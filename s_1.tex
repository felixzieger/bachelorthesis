\section{Extensions and Resolutions}
\label{s:extensionsandresolutions}
Throughout this thesis $R$ will denote a ring with identity.
We only consider left modules over rings.
If not stated otherwise, all modules are $R$-modules and all module homomorphisms are $R$-module homomorphisms.
The notation for an identity map of a set $X$ into itself is $1_X$.
Image and kernel of a map $\alpha$ are denoted $\Image \alpha$ and $\ker \alpha$ respectively.
\subsection{Ext}
\label{ss:ext1}
Let $A$ and $C$ be $R$-modules.
An extension $E$ of $A$ by $C$ is a short exact sequence
\begin{equation}
\label{shortexactsequence}
\begin{tikzcd} 
E: 0 \arrow{r} & A \arrow{r}{\chi} & B \arrow{r}{\sigma} & C \arrow{r}& 0
\end{tikzcd}
\end{equation}
of $R$-modules.
When speaking of extensions we always mean the associated modules and homomorphisms.
We write $E=(\chi,\sigma)$ for a sequence \eqref{shortexactsequence}.
A morphism $\Gamma$ from $E$ to $E'$ is a triple of module homomorphisms $(\alpha,\beta,\gamma)$ such that the diagram
\[ 
\begin{tikzcd}
E \arrow{d}{\Gamma}:& A \arrow{r} \arrow{d}{\alpha} & B \arrow{r} \arrow{d}{\beta} & C \arrow{d}{\gamma} \\
E': & A' \arrow{r} & B' \arrow{r} & C'
\end{tikzcd} 
\]
commutes.
For a morphism of the form 
\begin{equation} 
\label{diag:equivalence}
( 1_{A} , \beta , 1_{C} ):E \rightarrow E' 
\end{equation} 
the Short Five Lemma \cite[Lemma I.3.1.]{maclane} assures us that $\beta$ is an isomorphism.
Therefore the existence of a morphism \eqref{diag:equivalence} defines an equivalence relation which we denote $E\equiv E'$.
Define $\Ext{}_{R}(C,A)$ to be the set of congruence classes of extensions of $A$ by $C$.
Given an extension $E$ as in \eqref{shortexactsequence} and a module homomorphism $\alpha:A\to A'$, we can construct an extension $\alpha E$ of $A'$ by $C$ and a morphism $\Gamma=(\alpha, \beta, 1_C): E \to \alpha E$ as the push out along $\alpha$.
We can also pull back along a module homomorphism $\gamma: C' \to C$ and thereby construct an extension $E \gamma$ of $A$ by $C'$ and a morphism $\Gamma_1=(1_A,\beta_1, \gamma): E\gamma \to E$.
The pairs $(\Gamma, \alpha E)$ and $(\Gamma_1, E \gamma)$ are unique up to congruence.
For details on how to push out, pull back and a proof see \cite[Chapter III.1.]{maclane}.
We denote the category of left $R$-modules by $\mathbf{M}$  and the category of abelian groups by $\mathbf {Ab}$.
The bifunctor $\Ext{}_{R}(C,A)$ from $\mathbf {M \times M}$ to $\mathbf {Ab}$ is contravariant in its first and covariant in its second argument \cite[see Chapters III.1. and III.2.]{maclane}.
The group operation on $\Ext{}(C,A)$ will be explained in subsection \ref{ss:addition_in_ext}.
\begin{definition}
Let $A$ and $C$ be modules.
We denote their direct sum by $A \oplus C$.
The short exact sequence
\[
\begin{tikzcd}
0 \arrow{r}&
A \arrow{r}&
A \oplus C \arrow{r}&
C \arrow{r}&
0
\end{tikzcd}
\]
is called direct sum extension of $A$ by $C$.
\end{definition}
An extension congruent to the direct sum extension is called split.
\begin{example}\cite[Exercise 3.4.1.]{weibel}
Let $p$ be prime and $\mathbb{Z}/p$ be the cyclic group of order $p$.
Every extension $E$ of $\mathbb{Z}/p$ by $\mathbb{Z}/p$ is either split or congruent to a sequence of the form
$0\to \mathbb{Z}/p\xrightarrow{\cdot p}\mathbb{Z}/{p^2}\xrightarrow{\cdot i}\mathbb{Z}/p\to 0$, for $i=1,\dotsc,p-1$.
\end{example}
\begin{proof}
Let $E = (\chi, \sigma) : 0 \to \mathbb{Z}/p \to B \to \mathbb{Z}/p \to 0$ be a short exact sequence.
Because $\frac{B}{\mathbb{Z}/p} \cong \mathbb{Z}/p$ the group $B$ must be of order $p^2$. %TODO reference
There are two groups of order $p^2$, the cyclic group $\mathbb{Z}/{p^2}$ and the direct sum $\mathbb{Z}/p \oplus \mathbb{Z}/p$. %TODO reference

If $B\cong \mathbb{Z}/p \oplus \mathbb{Z}/p$, the monomorphism $\chi$ must send the generator $s$ of $\mathbb{Z}/p$ to $(i,j)$ with $i,j\in \mathbb{Z}/p$ and not both zero.
Suppose $i\neq 0$ then $(i,0)\mapsto s$ is a left inverse of $\chi$ and by \cite[Proposition I.4.3.]{maclane} $E$ is isomorphic to the direct sum sequence.

If $B\cong \mathbb{Z}/{p^2}$, the monomorphism $\chi$ must send the generator to a multiple of $p$.
Then $\sigma$ can send the generator of $\mathbb{Z}/{p^2}$ to any integer from $1,\dotsc,p-1$.
Let $i$ and $j\in {1,\dots, p-1}$.
Notice that for $E=(ip,j)$ we have a congruence
\begin{equation*}
\begin{tikzcd}
E: &
\mathbb{Z}/p 
	\arrow{r}{\cdot ip} 
	\arrow[equal,d]&
\mathbb{Z}/p^2 
	\arrow{r}{\cdot j} 
	\arrow{d}{\cdot i^{-1}}&
\mathbb{Z}/p 
	\arrow[d,equal]
\\
E': &
\mathbb{Z}/p 
	\arrow{r}{\cdot p}&
\mathbb{Z}/p^2 
	\arrow{r}{\cdot ij}&
\mathbb{Z}/p
\end{tikzcd}
\end{equation*}
to $E' = (p,ij) = (p, ij \bmod p)$.
And two extensions $(p,i')$ and $(p,j')$ are congruent only if $i' = j'$.
\end{proof}
\begin{lemma}
Let $E$ be as in \eqref{shortexactsequence}.
Then $\chi E$ and $E \sigma$ are split extensions.
\end{lemma}
\begin{proof}
By uniqueness of $\chi E$ and $E\sigma$ up to congruence it is enough to push out and pull back the sequence $E$ to split extensions.
Let $\iota, \iota'$ denote inclusions into the first and $\rho, \rho'$ projections onto the second factor.
The diagrams
\begin{equation*}
\begin{tikzcd}
A \arrow{r}{\chi} \arrow{d}{\chi}&
B \arrow{r}{\sigma} \arrow{d}{(1_B,\sigma)}&
C \arrow[d,equal]
\\
B \arrow{r}{\iota} &
B \oplus C \arrow{r}{\rho} &
C
\end{tikzcd}
\quad \text{and} \quad
\begin{tikzcd}
A \arrow{r}{\iota'} \arrow[d,equal]&
A\oplus B \arrow{r}{\rho'} \arrow{d}{\chi + 1_B}&
B \arrow{d}{\sigma}
\\
A \arrow{r}{\chi} &
B \arrow{r}{\sigma} &
C 
\end{tikzcd}
\end{equation*}
are commutative.
\end{proof}
%\begin{lemma} %only needed for naturality of E_*
%\label{lemma:alphaEequivEgamma}
%Let $\Gamma$ be of the form \eqref{diag:equivalence}.
%Then $\alpha E \equiv E' \gamma$.
%\end{lemma}
%\begin{proof}
%See \cite[Proposition III.1.8.]{maclane}.
%\end{proof}
\subsection{Ext$^\text{n}$} 
%EXTENSION
Fix a positive integer $n$.
We call an exact sequence of modules
\begin{equation}
\label{nfoldextension}
S = (\chi, \lambda_{n-1}, \dots, \lambda_1, \sigma) : 0 \rightarrow A \rightarrow B_{n-1} \rightarrow \dots  \rightarrow B_0 \rightarrow C \rightarrow 0
\end{equation}
an $n$-fold extension of $A$ by $C$.
Let $S$ and $S'$ be $n$-fold extensions.
%MORPHISM
A morphism $\Gamma:S\rightarrow S'$ is a tuple of $n+2$ module homomorphisms $(\alpha,\beta_{n-1}...,\beta_{0},\gamma)$ such that the diagram
\[\begin{tikzcd}
S: \arrow{d} & 
%0 \arrow{r} & 
A \arrow{r} \arrow{d}{\alpha} &
B_{n-1} \arrow{r}\arrow{d}{\beta_{n-1}} & 
\dots \arrow{r} & 
%B_1 \arrow{r}\arrow{d} & 
B_0 \arrow{r} \arrow{d}{\beta_0}& 
C %\arrow{r} 
\arrow{d}{\gamma} 
%& 0 
\\
S': & 
%0 \arrow{r} & 
A' \arrow{r} & 
B_{n-1}' \arrow{r} & 
\dots \arrow{r} &
%B_1'\arrow{r} & 
B_0' \arrow{r} & 
C' %\arrow{r} & 
%0
\end{tikzcd}\]
commutes.
%CONGRUENCE
Let $S$ ad $S'$ be two $n$-fold extensions of $A$ by $C$.
We say $S$ is congruent to $S'$ if there is a positive integer $k$, modules $S_0,S_1 \dots,S_{2k}$ with $S=S_0$ and $S_{2k}=S'$ and morphisms
\begin{equation*}
%\label{nfoldcongruence}
S_0 \to S_1 \leftarrow S_2 \to S_3 \leftarrow \dots \to S_{2k-1} \leftarrow S_{2k}
\end{equation*}
all starting in $1_A$ and ending in $1_C$.
This defines an equivalence relation on the set of all $n$-fold extensions of $A$ by $C$.
Notice that the definition for congruence on $\Ext{}_{R}$ agrees with the one defined here for $n=1$.
%EXT^n
Let $\Ext{n}_{R}(C,A)$ denote the set of congruence classes of $n$-fold extensions of $A$ by $C$.
When the ring is clear we write $\Ext{n}(C,A)$.
A congruence class $\sigma\in\Ext{n}(C,A)$ consists of $n$-fold extensions of $A$ by $C$.
If we are not interested in the congruence class of an extension, we use the notation $S \in\in \Ext{n}(C,A)$ for an element $S\in\sigma\in \Ext{n}(C,A)$.
\subsubsection{Splicing and Factorizing}
%%%
%%% Splicing
%%%
Given exact sequences
\begin{gather*}
S :0 \to A \to B_{n-1}\to \dots\to B_0 \xrightarrow{\sigma} K \to 0
\\
S': 0 \to K \xrightarrow{\chi} B'_{m-1}\to \dots \to B'_0 \to C \to 0
\end{gather*}
we assign to them an exact sequence $S \circ S'$ defined as
\begin{equation*} 
\label{diag:factorize} 
0 \to A \to \dots \to B_0 \xrightarrow{\lambda} B'_{m-1} \to \dots \to C \to 0
\end{equation*}
where $\lambda := \chi\sigma$.
This process is called splicing and $S\circ S'$ is called the Yoneda composite of $S$ and $S'$.
Furthermore any exact sequence can be factorized into short exact sequences.
%%%
%%% FACTORIZATION
%%%
Let $S$ be as in \eqref{nfoldextension}.
Let $\rho$ be the restriction of $B_{n-1}\to B_{n-2}$ onto its image and $\iota: \ker(B_{n-2} \to B_{n-3}) \to B_{n-2}$ the inclusion.
Then $S =: S_n = E_n \circ S_{n-1}$ where
\begin{gather*}
E_n : 0 \to A \to B_{n-1} \xrightarrow{\rho}\Image(B_{n-1}\to B_{n-2}) \to 0 
\\
S_{n-1} : 0 \to {\ker(B_{n-2} \to B_{n-3})} \xrightarrow{\iota}B_{n-2} \to \dots \to B_0 \to C \to 0
\end{gather*}
are both exact.
Iterating this process for $S_{n-1}$ and so on gives us a factorization of $S$ into short exact sequences $S=E_n \circ \dots \circ E_1$ where $E_1 := S_1$.
\subsubsection{Ext$^{\text{n}}$ as Bifunctor}
\label{ss:extnasbifunctor}
Let $S$ be as in \eqref{nfoldextension} with factorization $S=E_n 
\circ \dots \circ E_1$.
Let $\alpha: A\to A'$ and $\gamma: C' \to C$ be module homomorphisms.
We assign extensions $\alpha S := (\alpha E_n) \circ \dots \circ E_1$ of $A'$ by $C$ and $S\gamma := E_n \circ \dots \circ (E_1 \gamma)$ of $A$ by $C'$.
By this definition we get morphisms $\Gamma: S\to \alpha S$ and $\Gamma_1: S\gamma \to S $.\footnote{Pulling back and pushing out gives morphisms between the short exact sequences as mentioned in subsection \ref{ss:ext1}. Use the identity everywhere else.}
%BIFUNCTOR
The bifunctor $\Ext{n}_{R}(C,A): \mathbf{M \times M} \to \mathbf {Ab}$ is contravariant in its first and covariant in its second argument \cite[p. 85]{maclane}.
The group operation is presented in the succeeding subsection.

We will later need
\begin{lemma}
\label{lemma:morphisminducescongruence_n}
Every morphism $\Gamma = (\alpha,\dots,\gamma) : S \to S'$ between two $n$-fold extensions $S$ and $S'$ yields a congruence $\alpha S \equiv S' \gamma$.
\end{lemma}
\begin{proof}
See \cite[Proposition III.5.1.]{maclane}.
\end{proof}
\subsubsection{Addition in Ext$^\text{n}$}
\label{ss:addition_in_ext}
We will sketch how to define an addition on $\Ext{n}_{R}$.
Define the diagonal
%\begin{equation*}
$\Delta:  C \to C \oplus C \quad \text{by} \quad c \mapsto (c,c)$
%\end{equation*}
and the codiagonal
%\begin{equation*}
$\nabla: A \oplus A \to A \quad \text{by} \quad (a,a_1) \mapsto a + a_1$.
%\end{equation*}
Given two $n$-fold extensions $S,S'$ of $A$ by $C$ we define $S \oplus S'$ to be the component wise direct sum of their modules and homomorphisms.
The resulting sequence is exact, starts in $A \oplus A$ and ends in $C \oplus C$.
We define $S + S'$ by $\nabla (S \oplus S') \Delta$.
Note that $(\nabla (S \oplus S')) \Delta = \nabla ((S \oplus S') \Delta)$ so we can omit the parentheses.
The element $S+S'$ is called the Baer sum of $S$ and $S'$.
Together with this operation $\Ext{n}_{R}(C,A)$ is an abelian group \cite[p. 85]{maclane}.
The congruence class of the sequence 
\begin{equation}
\label{nfoldsplitextension}
S_0 = (1_A,0,\dots,0,1_C): 0\to A\to A \to 0 \to \dots \to 0 \to C \to C \to 0
\end{equation} 
is the zero element of $\Ext{n}_{R}(C,A)$ under the Baer sum.
Let $\alpha: A \to A'$, $\gamma: C' \to C$ be homomorphisms and $S,S' \in \in \Ext{n}_{R}(C,A)$.
The Baer sum is distributive \cite[see Theorem 5.3.]{maclane}.
That is:
\begin{equation*}
\alpha (S + S') \equiv \alpha S + \alpha S', \quad
(S + S')\gamma \equiv S\gamma + S' \gamma
\end{equation*}
We will later need
\begin{lemma}
\label{lemma:nfoldcomposite}
If $S$ is as in \eqref{nfoldextension}, the composite extensions $\chi S$ and $S \sigma$ are congruent to the zero element $S_0$ as defined in \eqref{nfoldsplitextension}.
\end{lemma}
\begin{proof}
Consider the morphisms $S \to \chi S \to S_0$ in the following diagram
\begin{equation*}
\begin{tikzcd}
A \arrow{r}{\chi} \arrow{d}{\chi} &
B_{n-1} \arrow{r}{\lambda_{n-1}}\arrow{d}{(1,\lambda_{n-1})} &
B_{n-2} \arrow{r}\arrow[d,equal] &
\dots \arrow{r} & 
%B_1 \arrow{r}\arrow[d,equal] & 
B_0 \arrow{r} \arrow[d,equal]& 
C  \arrow[d,equal]
\\ 
B_{n-1} \arrow{r} \arrow[d, equal] &
B_{n-1} \oplus \lambda_{n-1} B_{n-1} \arrow{r}{\rho_2}\arrow{d}{\rho_1} &
B_{n-2} \arrow{r}\arrow{d} &
\dots \arrow{r} & 
%B_1 \arrow{r}\arrow{d} & 
B_0 \arrow{r} \arrow{d}& 
C  \arrow[d,equal]
\\
B_{n-1} \arrow{r}{1} & 
B_{n-1} \arrow{r} & 
0 \arrow{r} &
\dots \arrow{r} &
%0 \arrow{r} & 
C \arrow{r}{1} & 
C 
\end{tikzcd}
\end{equation*}
where $\rho_1$ is the projection onto the first and $\rho_2$ the projection onto the second factor.
This shows that $\chi S \equiv S_0$.
One proves $S\sigma \equiv S_0$ via construction of morphisms $S_0 \leftarrow S\sigma \to S$ in a similar fashion.
\end{proof}
\subsection{Resolutions}
We recall some basic definitions from homological algebra.
\begin{definition}
A module $P$ is projective if for all epimorphisms $\alpha: B \twoheadrightarrow C$ and every module homomorphism $\gamma: P \to C$ there is a homomorphism $\beta: P \to B$ such that $\alpha \beta = \gamma$.
\end{definition}
\begin{definition}
A chain complex $X$ is a family $(X_n,\bdd{n})_{n\in \mathbb{Z}}$ of modules $X_n$ and homomorphisms $\bdd{n}:X_n\to X_{n-1}$ satisfying $\bdd{n}\bdd{n+1} = 0$ for all $n\in \mathbb{Z}$.
\end{definition}
\begin{definition}
The homology $H(X)$ of a complex $X=(X_n,\bdd{n})$ is the family of modules $H_n(X):=\ker\bdd{n}/\Image \bdd{n+1}$.
\end{definition}
Let $C$ be a module, $X = (X_n, \bdd{n})$ a chain complex, trivial in negative degrees, and $\epsilon:X_0 \rightarrow C$ a module homomorphism with $\epsilon \bdd{1}=0$.
The pair $(X,\epsilon)$ is called a complex over $C$.
If $X$ has trivial homology $H_n(X)$ for every $n>0$ and $\bdd{1}X_1 = \ker \epsilon$, then $(X,\epsilon)$ is called a resolution of $C$.
%free and projective resolution and complex over C
%If all $X_n$ are free modules then $(X,\epsilon)$ is called free.
If all $X_n$ are projective modules, then $(X,\epsilon)$ is called projective.
%Similarly $(X,\epsilon)$ is called projective if all $X_n$ are projective.
\begin{definition}
Let $X = (X_n, \bdd{n})$ and $X'=(X'_n, \bdd{n}')$ be chain complexes. A chain transformation $f:X\to X'$ is a family of module homomorphisms $f_n: X_n \to X_n'$ satisfying $\bdd{n}'f_n = f_{n-1}\bdd{n}$ for all $n\in \mathbb{Z}$.
\end{definition}
\begin{theorem}\cite[Theorem II.6.1.]{maclane}
\label{thm:comparison}
Let $C$ and $C'$ be modules.
Given a projective complex $(X,\epsilon)$ over $C$, a resolution $(X',\epsilon')$ of $C'$ and a module homomorphism $\gamma: C\to C'$ there is a chain transformation $f:X\to X'$ lifting $\gamma$.
That is, there are homomorphisms $f_0, f_1, \dots$ such that the diagram
\[
\begin{tikzcd}
 \dots \arrow{r}& 
 X_2 \arrow{r}{\bdd{2}} \arrow[dotted]{d}{f_2}&
 X_1 \arrow{r}{\bdd{1}} \arrow[dotted]{d}{f_1}&
 X_0 \arrow{r}{\epsilon} \arrow[dotted]{d}{f_0}& 
 C \arrow{d}{\gamma} \\
 \dots \arrow{r} &
 X_2' \arrow{r}{\bdd{2}'} &
 X_1' \arrow{r}{\bdd{1}'} &
 X_0' \arrow[two heads]{r}{\epsilon'} &
 C'
\end{tikzcd}
\]
commutes.
\end{theorem}
\begin{proof}
%Edge case f_0
Because $X_0$ is projective and $\epsilon': X'_0 \to C'$ is surjective there is a map $f_0:X_0\to X'_0$ with $\epsilon' f_0 = \gamma \epsilon$.
%Induction step
Suppose we constructed maps $f_0, \dots, f_n$.
Because $\bdd{n}\bdd{n+1} = 0$ and $f_{n-1}\bdd{n}= \bdd{n}'f_n$ we have $0 = f_{n-1}\bdd{n}\bdd{n+1} = \bdd{n}'f_n\bdd{n+1}$.
Hence $f_n \bdd{n+1}X_{n+1} \subset \ker\bdd{n}' = \bdd{n+1}'X_{n+1}'$ by exactness at $X_{n}'$.
As $X_{n+1}$ is projective there is a map $f_{n+1}$ with $\bdd{n+1}' f_{n+1} = f_n \bdd{n+1}$.
\end{proof}
\begin{definition}
Let $f,f':X\to X'$ be chain transformations.
A chain homotopy $s$ between $f$ and $f'$ is a familiy of maps $s_n:X_n\to X_{n+1}'$ satisfying $f_n - f'_n = s_{n-1}\bdd{n} + \bdd{n+1}'s_n$ for all $n \in \mathbb{Z}$.
\end{definition}
\begin{lemma}
\label{lemma:comparison}
Under the assumptions of Theorem \ref{thm:comparison}, two chain transformations $f,f'$ lifting the same $\gamma$ are chain homotopic.
\end{lemma}
\begin{proof}
For convenience we use the same symbol $\bdd{}$ for all boundary maps $\bdd{n}$ and $\bdd{n}'$.
We want to construct maps $s_n : X_n \to X'_{n+1}$ satisfying
\begin{equation}
\label{eq:homotopyatzero}
f_0 - f'_0 = \bdd{} s_0
\end{equation}
\begin{equation}
\label{eq:homotopyatnplusone}
f_{n+1} - f'_{n+1} = \bdd{} s_{n+1} + s_n \bdd{}
\end{equation}
for $n\geq 0$.
%Deal with edge case t, s_0
We observe that by commutativity $\epsilon'(f_0 - f'_0)=0$.
By exactness of the bottom row we know $(f_0 - f'_0)X_0 \subset \bdd{} X'_1$.
Projectivity of $X_0$ gives us a map $s_0:X_0 \to X'_1$ satisfying \eqref{eq:homotopyatzero}.
%Induction step
Suppose we have maps $s_0,\dots,s_n$ satisfying \eqref{eq:homotopyatzero} and \eqref{eq:homotopyatnplusone}.
Then 
$\bdd{} s_n = f_n - f'_n - s_{n-1} \bdd{}$
and therefore
$\bdd{}(f_{n+1} - f'_{n+1} - s_n \bdd{}) = ( f_n - f'_n )\bdd{} - (f_n - f'_n -s_{n-1}\bdd{})\bdd{} = 0$.
Exactness of the bottom row implies 
$(f_{n+1} - f'_{n+1} - s_n \bdd{})X_{n+1} \subset \bdd{} X'_{n+2}$.
Finally, because $X_{n+1}$ is projective, we get a map 
$s_{n+1}: X_{n+1} \to X'_{n+2}$
satisfying
$\bdd{} s_{n+1} = f_{n+1} - f'_{n+1} - s_n\bdd{}$.
\end{proof}
\subsection{Ext$^\text{n}$ and Resolutions}
\label{ss:extnandresolutions}
We have to recall some more definitions.
\begin{definition}
Let $G$ be a module and $X=(X_n,\bdd{n})$ a complex.
The cohomology $H^*(X,G)$ of $X$ with coefficients in $G$ is defined as the homology of the complex $\Hom(X,G)$.
An element of $\Hom(X_n,G)$ is called $n$-cochain.
The coboundary for an $n$-cochain $f$ is defined by $\delta^n f = (-1)^{n+1}f\bdd{n+1}$.
An element of $\ker \delta^n$ is called $n$-cocycle.
Two $n$-cocycles are cohomologous if their difference is the coboundary of an ($n$-1)-cochain.
\end{definition}

For $n>0$ the groups $\Ext{n}(C,A)$ can be computed as cohomology groups $H^{n}(X,A)$ of a projective resolution of $C$.

Regard an $n$-fold extension $S$ of $A$ by $C$ as a resolution of $C$ with zeros beyond the $n$th term.
By Theorem \ref{thm:comparison} we can lift the identity $1_C$ to a chain transformation $g:X\to S$:
\[
\begin{tikzcd}
X_{n+1} \arrow{r}{\bdd{n+1}} & 
X_n \arrow{r}{\bdd{n}} \arrow[dotted]{d}{g_n} & 
X_{n-1}\arrow{r} \arrow[dotted]{d}{g_{n-1}}&
\dots \arrow{r} & 
X_0 \arrow{r} \arrow[dotted]{d}{g_{0}}& 
C \arrow[d,equal]
\\
0 \arrow{r} &
A \arrow{r} & 
B_{n-1} \arrow{r} &
\dots \arrow{r} &
B_0 \arrow{r} &
C
\end{tikzcd}
\]
Note that by commutativity $g_n \bdd{n+1} = 0$.
This means that $g_n$ is a cocycle.
Define a map $\zeta: \Ext{n}(C,A) \to H^{n}(X,A)$ by $\zeta( \cls(S)) := \cls(g_n)$, where $\cls$ assigns a congruence class to an extension (or a cohomology class in the case of a cocycle) to the respective element.
Now we show that $\zeta$ is well defined.
\begin{enumerate}
\item \label{chainhomotopicyieldscohomologous elements}
We need to show that any two chain transformations lifting $1_C$ yield cohomologous elements.
Suppose $g':X\to S$ is a seconds chain transformation lifting $1_C$.
Let $s$ denote a chain homotopy between $g$ and $g'$ given by Lemma \ref{lemma:comparison}.
Note that $s_n: X_n \to 0$ is zero.
The chain homotopy there gives $g_n-g'_n= s_{n-1} \bdd{n}$, so $g_n$ and $g'_n$ are cohomologous.
\item
We need to prove that the class of $g_n$ does not depend on the representative of $\cls(S)$.
By our definition of congruence of $n$-fold exact sequences it is sufficient to check two cases.

First case.
Suppose $\Gamma: S\to S'$ is a morphism that starts with $1_A$  and ends with $1_C$.
Then $(\Gamma g)_n = g_n$ and $\Gamma g$ is a chain transformation.

Second case.
Suppose $\Gamma: S'' \to S$ is a morphism that starts with $1_A$ and ends with $1_C$. 
Now we construct a chain transformation $f: X \to S''$ lifting $1_C$ as in Theorem \ref{thm:zeta}.
Then $\Gamma f$ and $g$ are chain transformations lifting $1_C$.
We saw in \ref{chainhomotopicyieldscohomologous elements} that they therefore yield cohomologous elements $f_n$ and $g_n$.
\end{enumerate}
\begin{theorem}\cite[Theorem III.6.4.]{maclane}
\label{thm:zeta}
Let $A$, $C$ be modules and
$(X,\epsilon)$
be a projective resolution of $C$.
Then the map
$
\zeta: \Ext{n}(C,A) \to H^{n}(X,A)
$
is an isomorphism for $n$>0.
$\zeta$ is natural in $A$.
\end{theorem}
\begin{proof}
The function $\eta$ defined below will be the inverse of $\zeta$.
We factor $\bdd{n}:X_n \to X_{n-1}$ as $(\bdd{}',\chi): X_n \to \bdd{n}X_n \to X_{n-1}$ with $\chi$ the inclusion.
Let $g: X_n \to A$ be an $n$-cocyle, i.e. $g \bdd{n+1} = 0$.
Because $\ker (\bdd{}') = \ker(\bdd{n}) = \bdd{n+1}X_{n+1} \subset ker(g)$ we can factor $g$ as $h\bdd{}'$ by the universal property of the quotient.
\begin{equation*}
\begin{tikzcd}
&
X_{n+1} \arrow{r}&
X_n \arrow{d}{\bdd{}'} \arrow{rd}{\bdd{n}}
\\
S_n:&
0 \arrow{r} &
\bdd{}X_n\arrow{r}{\chi} \arrow{d}{h}&
X_{n-1} \arrow{r} \arrow{d}&
\dots \arrow{r}&
%X_0 \arrow{r}&
C \arrow[d,equal]
\\
hS_n:&
0 \arrow{r}&
A \arrow{r}
&
B_{n-1} \arrow{r}&
\dots \arrow{r}&
%&
C
\end{tikzcd}
\end{equation*}
In the above diagram from \cite[p. 89]{maclane} $S_n$ is an $n$-fold exact sequence and $hS_n$ is the push out along $h$.
We define $\eta:H^n(X,A)\to \Ext{n}(C,A)$ by $\eta \cls(g) := \cls(hS_n)$.
Using the distributive law in $\Ext{}$ we show that $\eta$ is well defined.
Consider a coboundary $h\bdd{}'$ with $\delta k = h\bdd{}'$ for some $k:X_{n-1}\to A$.
By definition of the coboundary
$\delta k = (-1)^n k \bdd{} = (-1)^n k \chi\bdd{}'$.
Therefore $h = (-1)^n k \chi$ and $h S_n = ((-1)^n k \chi)S_n$.
By Lemma \ref{lemma:nfoldcomposite} the composite extension $\chi S_n$ is congruent to $ S_0$ as defined in \eqref{nfoldsplitextension}.
Because $S_0$ is the zero element of addition in $\Ext{n}$ the distributive law implies that cohomologous elements are assigned the same element.
This shows that $\eta$ is well defined.
Now we show that the maps are inverses of each other.

%$\eta\zeta = 1$
Let $S\in\in \Ext{n}(C,A)$ be an extension and $g:X \to S$ a chain transformation lifting $1_C$.
Denote the factorization of $g_n$ by  $h \bdd{}'$.
Notice that $(h, g_{n-1},\dots,g_0,1_C):S_n \to S$ is a morphism.
Lemma \ref{lemma:morphisminducescongruence_n} implies $hS_n \equiv S$.
This shows $\eta\zeta = 1$.

%$\zeta\eta = 1$
Consider a cocycle $g:X_n \to A$ with factorization $g = \bdd{}'h$. 
Constructing the sequence $hS_n$ yields a chain transformation $X \to hS_n$ via composition of $X\to S_n \to hS_n$.
The $n$th homomorphism of this chain transformation is exactly $g$.
Therefore $\zeta\eta = 1$.

%Proof $\zeta$ is natural in $A$
It still remains to prove that $\zeta$ is natural.
Let $\alpha: A \to A'$ be a module homomorphism.
We need to show that the diagram
\begin{equation*}
\label{diag:zetanatural}
\begin{tikzcd}
\Ext{n}(C,\_)(A) 
	\arrow{r}{\zeta} 
	\arrow{d}{\alpha_*}&
H^n(X,\_)(A)
	\arrow{d}{\alpha_*}
\\
\Ext{n}(C,\_)(A') 
	\arrow{r}{\zeta} &
H^n(X,\_)(A')
\end{tikzcd}
\end{equation*}
commutes.

Suppose $S\in \in \Ext{n}(C,A)$.

Let $g: X \to S$ be a chain transformation lifting $1_C$.
Then $\alpha_* \zeta \cls(S) = \cls(\alpha g_n)$.

Let $h: X \to \alpha S$ be a chain transformation lifting $1_C$.
Then $\zeta \alpha_* \cls (S) = \zeta \cls(\alpha S) = \cls(h_n)$.

By definition of $ \alpha S$ we have a morphism 
$\Gamma: S \to \alpha S$.
The composition $\Gamma g : X \to \alpha S$ also lifts $1_C$.
Thus $\Gamma g$ and $h$ are chain homotopic by Lemma \ref{lemma:comparison}, so $\cls(h_n) = cls((\Gamma g)_n) = cls(\alpha g_n)$.
\end{proof}
Theorem \ref{thm:zeta} is a useful tool for computing Ext groups.
\begin{application}
Let $A$ be an abelian group and $n,m$ positive integers.
Then 
\[\Ext{m}_{\mathbb{Z}}(\mathbb{Z}/n,A) \cong \begin{cases}
A/nA, & m = 1\\
0, & m\geq 2
\end{cases}\]
\end{application}
\begin{proof}
Let $\rho: \mathbb{Z} \to \mathbb{Z}/n$ be the projection.
Consider the exact sequence
\[
\begin{tikzcd}
(X,\rho): &
\dots \arrow{r}&
0 \arrow{r} &
0 \arrow{r} &
\mathbb{Z} \arrow{r}{\cdot n} &
\mathbb{Z} \arrow{r}{\rho} &
\mathbb{Z}/n \arrow{r} &
0.
\end{tikzcd}
\]
Because $\mathbb{Z}$ is a projective module over itself, the pair $(X,\rho)$ is a projective resolution of $\mathbb{Z}/n$.
Theorem \ref{thm:zeta} implies $\Ext{m}(\mathbb{Z}/n,A) \cong H^{m}(X,A)$ for $m >0$.
Denote the induced maps on $\Hom$-groups by $n_*$ and $p_*$.
We calculate
\[
\Image n_* = \left\{nf:\mathbb{Z}\to A|f\in \Hom(\mathbb{Z},A)\right\} \cong \left\{na|a\in A\right\} =: nA.
\]
This gives us
\begin{align*}
\Ext{1}(\mathbb{Z}/n,A) \cong \ker (\Hom(X_1,A) \to 0) / \Image n_* = A / nA
\end{align*}
and clearly the cohomology groups $H^m(X,A)$ are trivial for $m>2$.
\end{proof}
From here on we will denote $\Hom(C,A)$ by $\Ext{0}(C,A)$.
Given a resolution $\dots \to X_1 \to X_0 \xrightarrow{\epsilon} C \to 0$, the induced sequence
$\Hom(X_1,A) \leftarrow \Hom(X_0,A) \xleftarrow{\epsilon^*} \Hom(C,A) \leftarrow 0$
is exact \cite[Theorem II.6.1.]{maclane}.
Hence $\ker (\Hom(X_0,A)\to \Hom(X_1,A)) \cong \Image \epsilon^*$ and therefore $\epsilon^*: \Hom(C,A) = \Ext{0}(C,A) \cong H^0(X,A)$.
For convenience we will denote this isomorphism by $\zeta$.
\subsection{A Long Exact Sequence for Ext$^\text{n}$}
\label{ss:alongexactsequenceforextn}
%In our proof we will use a chain complex of free modules.:
\begin{definition}
A module $F$ is called free if it has a basis.
\end{definition}
\begin{lemma}
Every free module is projective.
\end{lemma}
\begin{proof}
Let $F$ be free with basis $T \subset F$.
Given an epimorphism $\alpha: B \twoheadrightarrow C$ and a homomorphism $\gamma: F \to C$, we can choose elements $b_t \in B$ with $\alpha b_t = \gamma t$ for each $t\in T$.
This defines a homomorphism $\beta: F \to B$ satisfying $\alpha\beta=\gamma$.
\end{proof}
In the main proof of this subsection we will use some notation of
\begin{theorem}\cite[Theorem II.4.5.]{maclane}
\label{thm:longcohomologysequence}
If $X$ is a projective complex of $R$-modules and if 
$E=(\chi,\sigma): 0 \to A \to B \to C \to 0$
is a short exact sequence of $R$-modules, there is a connecting homomorphism 
$\delta_E:H^n(X,C)\to H^{n+1}(X,A)$.
Explicitly $\delta_E$ is defined by
$\delta_E = \cls \chi^{-1}\delta\sigma^{-1}\cls^{-1}$ where $\delta$ is the coboundary.
The connecting homomorphism yields a long exact sequence
\[
\dots \to
H^n(K,A)\xrightarrow{\chi_*}
H^n(K,B)\xrightarrow{\sigma_*}
H^n(K,C) \xrightarrow{\delta_E}
H^{n+1}(K,A)\to
\dots.
\]
The maps $\chi_*$ and $\sigma_*$ are the induced maps on cohomology classes.
\end{theorem}
\begin{proof}
See \cite[Theorem II.4.5.]{maclane}.
\end{proof}
\begin{definition}
Given a short exact sequence $E$ from $A$ to $C$ we define the connecting homomorphisms
$E_*:\Ext{n}(G,C) \rightarrow \Ext{n+1}(G,A)$ for each $n\geq0$ by $E_*(cls(S)) := cls(E\circ S)$.
\end{definition}
To see that $E*$ is well defined, suppose we have representatives $S$ and $S'$ of an element $\sigma\in\Ext{n}(G,C)$.
By definition of congruence we have morphisms 
\[S \to S_1 \leftarrow S_2 \to \dots \to S_{2k-1} \leftarrow S'\]
all starting in $1_C$ and ending in $1_G$.
We use this sequence to construct morphisms
\[E\circ S \to E\circ S_1 \leftarrow E\circ  S_2 \to \dots \to E\circ  S_{2k-1} \leftarrow E\circ  S'\]
by filling up the missing module homomorphisms with identities.

We use the notation $E_*\tau = E\tau$ for  $\tau\in \Ext{n}(G,C)$.
We can regard $\Ext{n}(G)(E):=\Ext{n}(G,C)$ and $\Ext{n+1}(G)(E):=\Ext{n+1}(G,A)$ as covariant functors of $E$.
Then the connecting homomorphism is a natural transformation between $\Ext{n}(G)(\_)$ and $\Ext{n+1}(G)(\_)$ \cite[p. 97]{maclane}.

\begin{theorem}\cite[Theorem III.9.1.]{maclane}
Consider a short exact sequence $E = (\chi,\sigma): 0 \to A \to B \to C \to 0$ of $R$-modules and an $R$-module $G$.
Then
\[
\dots\to
\Ext{n}(G,A) \xrightarrow{\chi_*}
\Ext{n}(G,B) \xrightarrow{\sigma_*}
\Ext{n}(G,C) \xrightarrow{E_*}
\Ext{n+1}(G,A) 
\to\dots
\]
is exact.
It starts with $0\rightarrow \Ext{0}(G,A)$.
The maps involved are defined as
\begin{equation}
\label{eq:longexactexthomomorphisms}
\chi_*\rho = \chi \rho, \quad 
\sigma_*\omega = \sigma \omega, \quad
E_* \tau = E \tau
\end{equation}
for elements 
$\rho\in\Ext{n}(G,A)$,
$\omega\in\Ext{n}(G,B)$ 
and 
$\tau\in\Ext{n}(G,C)$.
\end{theorem}
\begin{proof}
Because every module is isomorphic to a quotient of a free module \cite[Proposition I.5.3.]{maclane} we can construct free resolutions for any module.\footnote{Suppose we have a module $A\cong F_0/G_0$ for $F_0$ free. Then $G_0\cong F_1/G_1$ with $F_1$ free and so on. Then $\dots \to F_1\to F_0 \to A \to 0$ is a free resolution of $A$.}
Let $X$ be a free resolution of $G$.
Theorems \ref{thm:zeta} and \ref{thm:longcohomologysequence} yield a long exact sequence for $\Ext{}$:
\begin{equation}
\begin{tikzcd}
\label{diag:longexactsequenceisoms}
%\dots\to %leave out dots to make diagram fit on page
\Ext{n}(G,A) \arrow{r}{\chi_*} \arrow{d}{\zeta}&
\Ext{n}(G,B) \arrow{r}{\sigma_*} \arrow{d}{\zeta}&
\Ext{n}(G,C) \arrow{r}{E_*} \arrow{d}{\zeta}&
\Ext{n+1}(G,A) \arrow{d}{\zeta}
%\to\dots 
\\
%\dots\to
H^{n}(X,A) \arrow{r}{\chi_*} &
H^{n}(X,B) \arrow{r}{\sigma_*} &
H^{n}(X,C) \arrow{r}{(-1)^{n+1}\delta_E} &
H^{n+1}(X,A) 
%\to\dots
\end{tikzcd}
\end{equation}
It suffices to check that the maps defined in \eqref{eq:longexactexthomomorphisms} make the diagram commutative for every $n \geq 0$.
Commutativity of the first two squares of \eqref{diag:longexactsequenceisoms} follows by naturality of $\zeta$ for $n > 0$ and by recalling the definition of the isomorphism for $n=0$.

To prove commutativity of the square on the right in \eqref{diag:longexactsequenceisoms} in the case $n=0$, we have to show that
$(-1)\delta_E \zeta = \zeta E_*$.
Let $\gamma:G \to C$ be a homomorphism and $(1_A,\beta,\gamma):E\gamma\to E$ a morphism.
The diagram 
\begin{equation*}
\begin{tikzcd}
X:\arrow{d}&
X_1 
	\arrow{r}{\bdd{}}
	\arrow{d}{f_1} &
X_0 
	\arrow{r}{\epsilon}
	\arrow{d}{f_0} &
G
	\arrow[equal]{d} 
\\
E\gamma:\arrow{d}&
A
	\arrow{r} 
	\arrow[equal]{d} &
B'
	\arrow{r} 
	\arrow{d}{\beta}&
G
	\arrow{d}{\gamma}
\\
E:&
A
	\arrow{r}{\chi} &
B
	\arrow{r}{\sigma} &
C
\end{tikzcd}
\end{equation*}
shows chain transformations $X\to E\gamma \to E$.
By definition of $\delta_E$ and commutativity
\begin{multline*}
\delta_E \zeta \gamma 
= \delta_E \cls(\gamma \epsilon)
= \cls \chi^{-1} \delta \sigma ^{-1} \gamma \epsilon
\\= \cls \chi^{-1} \delta (\beta f_0)
= (-1) \cls \chi^{-1} \beta f_0 \bdd{}
= (-1) \cls (f_1).
\end{multline*}
On the other hand $\zeta E_* \gamma = \zeta \cls(E\gamma) = cls(f_1)$.
Thus the case is proven.

%commutativity of square with $\delta_E$
Let $S \in\in \Ext{n}(G,C)$ for $n>0$.
Regard $E \circ S$ as a resolution of $G$.
Let $f:X\to E\circ S$ be a chain transformation lifting $1_G$.
The diagram
\begin{equation*}
%\label{diag:longexactsequenceproof}
\begin{tikzcd}
X: \arrow{d}{f}&
X_{n+1}
	\arrow{r}{\bdd{n+1}}
	\arrow{d}{f_{n+1}}&
X_{n}
	\arrow{r}
	\arrow{d}{f_n}&
X_{n-1}
	\arrow{r}
	\arrow{d}{f_{n-1}}&
\dots
	\arrow{r}&
X_0
	\arrow{r}
	\arrow{d}{f_0}&
G
	\arrow[d, equal]
\\
E\circ S:\arrow{d}{\Gamma}&
A
	\arrow{r}{\chi}&
B
	\arrow{r}{\lambda\sigma}
	\arrow{d}{\sigma}&
B_{n-1}
	\arrow{r}
	\arrow[d,equal]&
\dots
	\arrow{r}&
B_0
	\arrow{r}
	\arrow[d,equal]&
G
	\arrow[d, equal]
\\
S:&
&
C
	\arrow{r}{\lambda}&
B_{n-1}
	\arrow{r}&
\dots
	\arrow{r}&
B_0
	\arrow{r}&
G
\end{tikzcd}
\end{equation*}
shows chain transformations $X\to E\circ S\to S$.
%%%
%%% \zeta E_*
%%%
By definition of $\zeta$ we get $\zeta E_* \cls(S) = \cls(f_{n+1})$.
%%%
%%% \delta_E \zeta
%%%
Composing $\Gamma f$ results in a chain transformation $X \to S$ lifting $1_G$.
Therefore $\zeta(S) = \cls((\Gamma f)_n) = \cls(\sigma f_n)$.
Hence $\delta_E \zeta(S)= \cls \chi^{-1}\delta f_n = (-1)^{n+1} \cls \chi^{-1} f_n \bdd{n+1} = (-1)^{n+1} \cls (f_{n+1})$.
The last equation follows by commutativity of the upper left square in the diagram.
Signs cancel out with those in our exact sequence \eqref{diag:longexactsequenceisoms}.
This shows commutativity for $n>0$.
\end{proof}
