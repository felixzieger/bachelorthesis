\subsection{A Long Exact Sequence for Ext$^\text{n}$}
\label{ss:alongexactsequenceforextn}
%In our proof we will use a chain complex of free modules.:
\begin{definition}
A module $F$ is called free if it has a basis.
\end{definition}
\begin{lemma}
Every free module is projective.
\end{lemma}
\begin{proof}
Let $F$ be free with basis $T \subset F$.
Given an epimorphism $\alpha: B \twoheadrightarrow C$ and a homomorphism $\gamma: F \to C$, we can choose elements $b_t \in B$ with $\alpha b_t = \gamma t$ for each $t\in T$.
This defines a homomorphism $\beta: F \to B$ satisfying $\alpha\beta=\gamma$.
\end{proof}
In the main proof of this subsection we will use some notation of
\begin{theorem}\cite[Theorem II.4.5.]{maclane}
\label{thm:longcohomologysequence}
If $X$ is a projective complex of $R$-modules and if 
$E=(\chi,\sigma): 0 \to A \to B \to C \to 0$
is a short exact sequence of $R$-modules, there is a connecting homomorphism 
$\delta_E:H^n(X,C)\to H^{n+1}(X,A)$.
Explicitly $\delta_E$ is defined by
$\delta_E = \cls \chi^{-1}\delta\sigma^{-1}\cls^{-1}$ where $\delta$ is the coboundary.
The connecting homomorphism yields a long exact sequence
\[
\dots \to
H^n(K,A)\xrightarrow{\chi_*}
H^n(K,B)\xrightarrow{\sigma_*}
H^n(K,C) \xrightarrow{\delta_E}
H^{n+1}(K,A)\to
\dots.
\]
The maps $\chi_*$ and $\sigma_*$ are the induced maps on cohomology classes.
\end{theorem}
\begin{proof}
See \cite[Theorem II.4.5.]{maclane}.
\end{proof}
\begin{definition}
Given a short exact sequence $E$ from $A$ to $C$ we define the connecting homomorphisms
$E_*:\Ext{n}(G,C) \rightarrow \Ext{n+1}(G,A)$ for each $n\geq0$ by $E_*(cls(S)) := cls(E\circ S)$.
\end{definition}
To see that $E*$ is well defined, suppose we have representatives $S$ and $S'$ of an element $\sigma\in\Ext{n}(G,C)$.
By definition of congruence we have morphisms 
\[S \to S_1 \leftarrow S_2 \to \dots \to S_{2k-1} \leftarrow S'\]
all starting in $1_C$ and ending in $1_G$.
We use this sequence to construct morphisms
\[E\circ S \to E\circ S_1 \leftarrow E\circ  S_2 \to \dots \to E\circ  S_{2k-1} \leftarrow E\circ  S'\]
by filling up the missing module homomorphisms with identities.

We use the notation $E_*\tau = E\tau$ for  $\tau\in \Ext{n}(G,C)$.
We can regard $\Ext{n}(G)(E):=\Ext{n}(G,C)$ and $\Ext{n+1}(G)(E):=\Ext{n+1}(G,A)$ as covariant functors of $E$.
Then the connecting homomorphism is a natural transformation between $\Ext{n}(G)(\_)$ and $\Ext{n+1}(G)(\_)$ \cite[p. 97]{maclane}.

\begin{theorem}\cite[Theorem III.9.1.]{maclane}
Consider a short exact sequence $E = (\chi,\sigma): 0 \to A \to B \to C \to 0$ of $R$-modules and an $R$-module $G$.
Then
\[
\dots\to
\Ext{n}(G,A) \xrightarrow{\chi_*}
\Ext{n}(G,B) \xrightarrow{\sigma_*}
\Ext{n}(G,C) \xrightarrow{E_*}
\Ext{n+1}(G,A) 
\to\dots
\]
is exact.
It starts with $0\rightarrow \Ext{0}(G,A)$.
The maps involved are defined as
\begin{equation}
\label{eq:longexactexthomomorphisms}
\chi_*\rho = \chi \rho, \quad 
\sigma_*\omega = \sigma \omega, \quad
E_* \tau = E \tau
\end{equation}
for elements 
$\rho\in\Ext{n}(G,A)$,
$\omega\in\Ext{n}(G,B)$ 
and 
$\tau\in\Ext{n}(G,C)$.
\end{theorem}
\begin{proof}
Because every module is isomorphic to a quotient of a free module \cite[Proposition I.5.3.]{maclane} we can construct free resolutions for any module.\footnote{Suppose we have a module $A\cong F_0/G_0$ for $F_0$ free. Then $G_0\cong F_1/G_1$ with $F_1$ free and so on. Then $\dots \to F_1\to F_0 \to A \to 0$ is a free resolution of $A$.}
Let $X$ be a free resolution of $G$.
Theorems \ref{thm:zeta} and \ref{thm:longcohomologysequence} yield a long exact sequence for $\Ext{}$:
\begin{equation}
\begin{tikzcd}
\label{diag:longexactsequenceisoms}
%\dots\to %leave out dots to make diagram fit on page
\Ext{n}(G,A) \arrow{r}{\chi_*} \arrow{d}{\zeta}&
\Ext{n}(G,B) \arrow{r}{\sigma_*} \arrow{d}{\zeta}&
\Ext{n}(G,C) \arrow{r}{E_*} \arrow{d}{\zeta}&
\Ext{n+1}(G,A) \arrow{d}{\zeta}
%\to\dots 
\\
%\dots\to
H^{n}(X,A) \arrow{r}{\chi_*} &
H^{n}(X,B) \arrow{r}{\sigma_*} &
H^{n}(X,C) \arrow{r}{(-1)^{n+1}\delta_E} &
H^{n+1}(X,A) 
%\to\dots
\end{tikzcd}
\end{equation}
It suffices to check that the maps defined in \eqref{eq:longexactexthomomorphisms} make the diagram commutative for every $n \geq 0$.
Commutativity of the first two squares of \eqref{diag:longexactsequenceisoms} follows by naturality of $\zeta$ for $n > 0$ and by recalling the definition of the isomorphism for $n=0$.

To prove commutativity of the square on the right in \eqref{diag:longexactsequenceisoms} in the case $n=0$, we have to show that
$(-1)\delta_E \zeta = \zeta E_*$.
Let $\gamma:G \to C$ be a homomorphism and $(1_A,\beta,\gamma):E\gamma\to E$ a morphism.
The diagram 
\begin{equation*}
\begin{tikzcd}
X:\arrow{d}&
X_1 
	\arrow{r}{\bdd{}}
	\arrow{d}{f_1} &
X_0 
	\arrow{r}{\epsilon}
	\arrow{d}{f_0} &
G
	\arrow[equal]{d} 
\\
E\gamma:\arrow{d}&
A
	\arrow{r} 
	\arrow[equal]{d} &
B'
	\arrow{r} 
	\arrow{d}{\beta}&
G
	\arrow{d}{\gamma}
\\
E:&
A
	\arrow{r}{\chi} &
B
	\arrow{r}{\sigma} &
C
\end{tikzcd}
\end{equation*}
shows chain transformations $X\to E\gamma \to E$.
By definition of $\delta_E$ and commutativity
\begin{multline*}
\delta_E \zeta \gamma 
= \delta_E \cls(\gamma \epsilon)
= \cls \chi^{-1} \delta \sigma ^{-1} \gamma \epsilon
\\= \cls \chi^{-1} \delta (\beta f_0)
= (-1) \cls \chi^{-1} \beta f_0 \bdd{}
= (-1) \cls (f_1).
\end{multline*}
On the other hand $\zeta E_* \gamma = \zeta \cls(E\gamma) = cls(f_1)$.
Thus the case is proven.

%commutativity of square with $\delta_E$
Let $S \in\in \Ext{n}(G,C)$ for $n>0$.
Regard $E \circ S$ as a resolution of $G$.
Let $f:X\to E\circ S$ be a chain transformation lifting $1_G$.
The diagram
\begin{equation*}
%\label{diag:longexactsequenceproof}
\begin{tikzcd}
X: \arrow{d}{f}&
X_{n+1}
	\arrow{r}{\bdd{n+1}}
	\arrow{d}{f_{n+1}}&
X_{n}
	\arrow{r}
	\arrow{d}{f_n}&
X_{n-1}
	\arrow{r}
	\arrow{d}{f_{n-1}}&
\dots
	\arrow{r}&
X_0
	\arrow{r}
	\arrow{d}{f_0}&
G
	\arrow[d, equal]
\\
E\circ S:\arrow{d}{\Gamma}&
A
	\arrow{r}{\chi}&
B
	\arrow{r}{\lambda\sigma}
	\arrow{d}{\sigma}&
B_{n-1}
	\arrow{r}
	\arrow[d,equal]&
\dots
	\arrow{r}&
B_0
	\arrow{r}
	\arrow[d,equal]&
G
	\arrow[d, equal]
\\
S:&
&
C
	\arrow{r}{\lambda}&
B_{n-1}
	\arrow{r}&
\dots
	\arrow{r}&
B_0
	\arrow{r}&
G
\end{tikzcd}
\end{equation*}
shows chain transformations $X\to E\circ S\to S$.
%%%
%%% \zeta E_*
%%%
By definition of $\zeta$ we get $\zeta E_* \cls(S) = \cls(f_{n+1})$.
%%%
%%% \delta_E \zeta
%%%
Composing $\Gamma f$ results in a chain transformation $X \to S$ lifting $1_G$.
Therefore $\zeta(S) = \cls((\Gamma f)_n) = \cls(\sigma f_n)$.
Hence $\delta_E \zeta(S)= \cls \chi^{-1}\delta f_n = (-1)^{n+1} \cls \chi^{-1} f_n \bdd{n+1} = (-1)^{n+1} \cls (f_{n+1})$.
The last equation follows by commutativity of the upper left square in the diagram.
Signs cancel out with those in our exact sequence \eqref{diag:longexactsequenceisoms}.
This shows commutativity for $n>0$.
\end{proof}