\subsection{Resolutions}
We recall some basic definitions from homological algebra.
\begin{definition}
A module $P$ is projective if for all epimorphisms $\alpha: B \twoheadrightarrow C$ and every module homomorphism $\gamma: P \to C$ there is a homomorphism $\beta: P \to B$ such that $\alpha \beta = \gamma$.
\end{definition}
\begin{definition}
A chain complex $X$ is a family $(X_n,\bdd{n})_{n\in \mathbb{Z}}$ of modules $X_n$ and homomorphisms $\bdd{n}:X_n\to X_{n-1}$ satisfying $\bdd{n}\bdd{n+1} = 0$ for all $n\in \mathbb{Z}$.
\end{definition}
\begin{definition}
The homology $H(X)$ of a complex $X=(X_n,\bdd{n})$ is the family of modules $H_n(X):=\ker\bdd{n}/\Image \bdd{n+1}$.
\end{definition}
Let $C$ be a module, $X = (X_n, \bdd{n})$ a chain complex, trivial in negative degrees, and $\epsilon:X_0 \rightarrow C$ a module homomorphism with $\epsilon \bdd{1}=0$.
The pair $(X,\epsilon)$ is called a complex over $C$.
If $X$ has trivial homology $H_n(X)$ for every $n>0$ and $\bdd{1}X_1 = \ker \epsilon$, then $(X,\epsilon)$ is called a resolution of $C$.
%free and projective resolution and complex over C
%If all $X_n$ are free modules then $(X,\epsilon)$ is called free.
If all $X_n$ are projective modules, then $(X,\epsilon)$ is called projective.
%Similarly $(X,\epsilon)$ is called projective if all $X_n$ are projective.
\begin{definition}
Let $X = (X_n, \bdd{n})$ and $X'=(X'_n, \bdd{n}')$ be chain complexes. A chain transformation $f:X\to X'$ is a family of module homomorphisms $f_n: X_n \to X_n'$ satisfying $\bdd{n}'f_n = f_{n-1}\bdd{n}$ for all $n\in \mathbb{Z}$.
\end{definition}
\begin{theorem}\cite[Theorem II.6.1.]{maclane}
\label{thm:comparison}
Let $C$ and $C'$ be modules.
Given a projective complex $(X,\epsilon)$ over $C$, a resolution $(X',\epsilon')$ of $C'$ and a module homomorphism $\gamma: C\to C'$ there is a chain transformation $f:X\to X'$ lifting $\gamma$.
That is, there are homomorphisms $f_0, f_1, \dots$ such that the diagram
\[
\begin{tikzcd}
 \dots \arrow{r}& 
 X_2 \arrow{r}{\bdd{2}} \arrow[dotted]{d}{f_2}&
 X_1 \arrow{r}{\bdd{1}} \arrow[dotted]{d}{f_1}&
 X_0 \arrow{r}{\epsilon} \arrow[dotted]{d}{f_0}& 
 C \arrow{d}{\gamma} \\
 \dots \arrow{r} &
 X_2' \arrow{r}{\bdd{2}'} &
 X_1' \arrow{r}{\bdd{1}'} &
 X_0' \arrow[two heads]{r}{\epsilon'} &
 C'
\end{tikzcd}
\]
commutes.
\end{theorem}
\begin{proof}
%Edge case f_0
Because $X_0$ is projective and $\epsilon': X'_0 \to C'$ is surjective there is a map $f_0:X_0\to X'_0$ with $\epsilon' f_0 = \gamma \epsilon$.
%Induction step
Suppose we constructed maps $f_0, \dots, f_n$.
Because $\bdd{n}\bdd{n+1} = 0$ and $f_{n-1}\bdd{n}= \bdd{n}'f_n$ we have $0 = f_{n-1}\bdd{n}\bdd{n+1} = \bdd{n}'f_n\bdd{n+1}$.
Hence $f_n \bdd{n+1}X_{n+1} \subset \ker\bdd{n}' = \bdd{n+1}'X_{n+1}'$ by exactness at $X_{n}'$.
As $X_{n+1}$ is projective there is a map $f_{n+1}$ with $\bdd{n+1}' f_{n+1} = f_n \bdd{n+1}$.
\end{proof}
\begin{definition}
Let $f,f':X\to X'$ be chain transformations.
A chain homotopy $s$ between $f$ and $f'$ is a familiy of maps $s_n:X_n\to X_{n+1}'$ satisfying $f_n - f'_n = s_{n-1}\bdd{n} + \bdd{n+1}'s_n$ for all $n \in \mathbb{Z}$.
\end{definition}
\begin{lemma}
\label{lemma:comparison}
Under the assumptions of Theorem \ref{thm:comparison}, two chain transformations $f,f'$ lifting the same $\gamma$ are chain homotopic.
\end{lemma}
\begin{proof}
For convenience we use the same symbol $\bdd{}$ for all boundary maps $\bdd{n}$ and $\bdd{n}'$.
We want to construct maps $s_n : X_n \to X'_{n+1}$ satisfying
\begin{equation}
\label{eq:homotopyatzero}
f_0 - f'_0 = \bdd{} s_0
\end{equation}
\begin{equation}
\label{eq:homotopyatnplusone}
f_{n+1} - f'_{n+1} = \bdd{} s_{n+1} + s_n \bdd{}
\end{equation}
for $n\geq 0$.
%Deal with edge case t, s_0
We observe that by commutativity $\epsilon'(f_0 - f'_0)=0$.
By exactness of the bottom row we know $(f_0 - f'_0)X_0 \subset \bdd{} X'_1$.
Projectivity of $X_0$ gives us a map $s_0:X_0 \to X'_1$ satisfying \eqref{eq:homotopyatzero}.
%Induction step
Suppose we have maps $s_0,\dots,s_n$ satisfying \eqref{eq:homotopyatzero} and \eqref{eq:homotopyatnplusone}.
Then 
$\bdd{} s_n = f_n - f'_n - s_{n-1} \bdd{}$
and therefore
$\bdd{}(f_{n+1} - f'_{n+1} - s_n \bdd{}) = ( f_n - f'_n )\bdd{} - (f_n - f'_n -s_{n-1}\bdd{})\bdd{} = 0$.
Exactness of the bottom row implies 
$(f_{n+1} - f'_{n+1} - s_n \bdd{})X_{n+1} \subset \bdd{} X'_{n+2}$.
Finally, because $X_{n+1}$ is projective, we get a map 
$s_{n+1}: X_{n+1} \to X'_{n+2}$
satisfying
$\bdd{} s_{n+1} = f_{n+1} - f'_{n+1} - s_n\bdd{}$.
\end{proof}
\subsection{Ext$^\text{n}$ and Resolutions}
\label{ss:extnandresolutions}
We have to recall some more definitions.
\begin{definition}
Let $G$ be a module and $X=(X_n,\bdd{n})$ a complex.
The cohomology $H^*(X,G)$ of $X$ with coefficients in $G$ is defined as the homology of the complex $\Hom(X,G)$.
An element of $\Hom(X_n,G)$ is called $n$-cochain.
The coboundary for an $n$-cochain $f$ is defined by $\delta^n f = (-1)^{n+1}f\bdd{n+1}$.
An element of $\ker \delta^n$ is called $n$-cocycle.
Two $n$-cocycles are cohomologous if their difference is the coboundary of an ($n$-1)-cochain.
\end{definition}

For $n>0$ the groups $\Ext{n}(C,A)$ can be computed as cohomology groups $H^{n}(X,A)$ of a projective resolution of $C$.

Regard an $n$-fold extension $S$ of $A$ by $C$ as a resolution of $C$ with zeros beyond the $n$th term.
By Theorem \ref{thm:comparison} we can lift the identity $1_C$ to a chain transformation $g:X\to S$:
\[
\begin{tikzcd}
X_{n+1} \arrow{r}{\bdd{n+1}} & 
X_n \arrow{r}{\bdd{n}} \arrow[dotted]{d}{g_n} & 
X_{n-1}\arrow{r} \arrow[dotted]{d}{g_{n-1}}&
\dots \arrow{r} & 
X_0 \arrow{r} \arrow[dotted]{d}{g_{0}}& 
C \arrow[d,equal]
\\
0 \arrow{r} &
A \arrow{r} & 
B_{n-1} \arrow{r} &
\dots \arrow{r} &
B_0 \arrow{r} &
C
\end{tikzcd}
\]
Note that by commutativity $g_n \bdd{n+1} = 0$.
This means that $g_n$ is a cocycle.
Define a map $\zeta: \Ext{n}(C,A) \to H^{n}(X,A)$ by $\zeta( \cls(S)) := \cls(g_n)$, where $\cls$ assigns a congruence class to an extension (or a cohomology class in the case of a cocycle) to the respective element.
Now we show that $\zeta$ is well defined.
\begin{enumerate}
\item \label{chainhomotopicyieldscohomologous elements}
We need to show that any two chain transformations lifting $1_C$ yield cohomologous elements.
Suppose $g':X\to S$ is a seconds chain transformation lifting $1_C$.
Let $s$ denote a chain homotopy between $g$ and $g'$ given by Lemma \ref{lemma:comparison}.
Note that $s_n: X_n \to 0$ is zero.
The chain homotopy there gives $g_n-g'_n= s_{n-1} \bdd{n}$, so $g_n$ and $g'_n$ are cohomologous.
\item
We need to prove that the class of $g_n$ does not depend on the representative of $\cls(S)$.
By our definition of congruence of $n$-fold exact sequences it is sufficient to check two cases.

First case.
Suppose $\Gamma: S\to S'$ is a morphism that starts with $1_A$  and ends with $1_C$.
Then $(\Gamma g)_n = g_n$ and $\Gamma g$ is a chain transformation.

Second case.
Suppose $\Gamma: S'' \to S$ is a morphism that starts with $1_A$ and ends with $1_C$. 
Now we construct a chain transformation $f: X \to S''$ lifting $1_C$ as in Theorem \ref{thm:zeta}.
Then $\Gamma f$ and $g$ are chain transformations lifting $1_C$.
We saw in \ref{chainhomotopicyieldscohomologous elements} that they therefore yield cohomologous elements $f_n$ and $g_n$.
\end{enumerate}
\begin{theorem}\cite[Theorem III.6.4.]{maclane}
\label{thm:zeta}
Let $A$, $C$ be modules and
$(X,\epsilon)$
be a projective resolution of $C$.
Then the map
$
\zeta: \Ext{n}(C,A) \to H^{n}(X,A)
$
is an isomorphism for $n$>0.
$\zeta$ is natural in $A$.
\end{theorem}
\begin{proof}
The function $\eta$ defined below will be the inverse of $\zeta$.
We factor $\bdd{n}:X_n \to X_{n-1}$ as $(\bdd{}',\chi): X_n \to \bdd{n}X_n \to X_{n-1}$ with $\chi$ the inclusion.
Let $g: X_n \to A$ be an $n$-cocyle, i.e. $g \bdd{n+1} = 0$.
Because $\ker (\bdd{}') = \ker(\bdd{n}) = \bdd{n+1}X_{n+1} \subset ker(g)$ we can factor $g$ as $h\bdd{}'$ by the universal property of the quotient.
\begin{equation*}
\begin{tikzcd}
&
X_{n+1} \arrow{r}&
X_n \arrow{d}{\bdd{}'} \arrow{rd}{\bdd{n}}
\\
S_n:&
0 \arrow{r} &
\bdd{}X_n\arrow{r}{\chi} \arrow{d}{h}&
X_{n-1} \arrow{r} \arrow{d}&
\dots \arrow{r}&
%X_0 \arrow{r}&
C \arrow[d,equal]
\\
hS_n:&
0 \arrow{r}&
A \arrow{r}
&
B_{n-1} \arrow{r}&
\dots \arrow{r}&
%&
C
\end{tikzcd}
\end{equation*}
In the above diagram from \cite[p. 89]{maclane} $S_n$ is an $n$-fold exact sequence and $hS_n$ is the push out along $h$.
We define $\eta:H^n(X,A)\to \Ext{n}(C,A)$ by $\eta \cls(g) := \cls(hS_n)$.
Using the distributive law in $\Ext{}$ we show that $\eta$ is well defined.
Consider a coboundary $h\bdd{}'$ with $\delta k = h\bdd{}'$ for some $k:X_{n-1}\to A$.
By definition of the coboundary
$\delta k = (-1)^n k \bdd{} = (-1)^n k \chi\bdd{}'$.
Therefore $h = (-1)^n k \chi$ and $h S_n = ((-1)^n k \chi)S_n$.
By Lemma \ref{lemma:nfoldcomposite} the composite extension $\chi S_n$ is congruent to $ S_0$ as defined in \eqref{nfoldsplitextension}.
Because $S_0$ is the zero element of addition in $\Ext{n}$ the distributive law implies that cohomologous elements are assigned the same element.
This shows that $\eta$ is well defined.
Now we show that the maps are inverses of each other.

%$\eta\zeta = 1$
Let $S\in\in \Ext{n}(C,A)$ be an extension and $g:X \to S$ a chain transformation lifting $1_C$.
Denote the factorization of $g_n$ by  $h \bdd{}'$.
Notice that $(h, g_{n-1},\dots,g_0,1_C):S_n \to S$ is a morphism.
Lemma \ref{lemma:morphisminducescongruence_n} implies $hS_n \equiv S$.
This shows $\eta\zeta = 1$.

%$\zeta\eta = 1$
Consider a cocycle $g:X_n \to A$ with factorization $g = \bdd{}'h$. 
Constructing the sequence $hS_n$ yields a chain transformation $X \to hS_n$ via composition of $X\to S_n \to hS_n$.
The $n$th homomorphism of this chain transformation is exactly $g$.
Therefore $\zeta\eta = 1$.

%Proof $\zeta$ is natural in $A$
It still remains to prove that $\zeta$ is natural.
Let $\alpha: A \to A'$ be a module homomorphism.
We need to show that the diagram
\begin{equation*}
\label{diag:zetanatural}
\begin{tikzcd}
\Ext{n}(C,\_)(A) 
	\arrow{r}{\zeta} 
	\arrow{d}{\alpha_*}&
H^n(X,\_)(A)
	\arrow{d}{\alpha_*}
\\
\Ext{n}(C,\_)(A') 
	\arrow{r}{\zeta} &
H^n(X,\_)(A')
\end{tikzcd}
\end{equation*}
commutes.

Suppose $S\in \in \Ext{n}(C,A)$.

Let $g: X \to S$ be a chain transformation lifting $1_C$.
Then $\alpha_* \zeta \cls(S) = \cls(\alpha g_n)$.

Let $h: X \to \alpha S$ be a chain transformation lifting $1_C$.
Then $\zeta \alpha_* \cls (S) = \zeta \cls(\alpha S) = \cls(h_n)$.

By definition of $ \alpha S$ we have a morphism 
$\Gamma: S \to \alpha S$.
The composition $\Gamma g : X \to \alpha S$ also lifts $1_C$.
Thus $\Gamma g$ and $h$ are chain homotopic by Lemma \ref{lemma:comparison}, so $\cls(h_n) = cls((\Gamma g)_n) = cls(\alpha g_n)$.
\end{proof}
Theorem \ref{thm:zeta} is a useful tool for computing Ext groups.
\begin{application}
Let $A$ be an abelian group and $n,m$ positive integers.
Then 
\[\Ext{m}_{\mathbb{Z}}(\mathbb{Z}/n,A) \cong \begin{cases}
A/nA, & m = 1\\
0, & m\geq 2
\end{cases}\]
\end{application}
\begin{proof}
Let $\rho: \mathbb{Z} \to \mathbb{Z}/n$ be the projection.
Consider the exact sequence
\[
\begin{tikzcd}
(X,\rho): &
\dots \arrow{r}&
0 \arrow{r} &
0 \arrow{r} &
\mathbb{Z} \arrow{r}{\cdot n} &
\mathbb{Z} \arrow{r}{\rho} &
\mathbb{Z}/n \arrow{r} &
0.
\end{tikzcd}
\]
Because $\mathbb{Z}$ is a projective module over itself, the pair $(X,\rho)$ is a projective resolution of $\mathbb{Z}/n$.
Theorem \ref{thm:zeta} implies $\Ext{m}(\mathbb{Z}/n,A) \cong H^{m}(X,A)$ for $m >0$.
Denote the induced maps on $\Hom$-groups by $n_*$ and $p_*$.
We calculate
\[
\Image n_* = \left\{nf:\mathbb{Z}\to A|f\in \Hom(\mathbb{Z},A)\right\} \cong \left\{na|a\in A\right\} =: nA.
\]
This gives us
\begin{align*}
\Ext{1}(\mathbb{Z}/n,A) \cong \ker (\Hom(X_1,A) \to 0) / \Image n_* = A / nA
\end{align*}
and clearly the cohomology groups $H^m(X,A)$ are trivial for $m>2$.
\end{proof}
From here on we will denote $\Hom(C,A)$ by $\Ext{0}(C,A)$.
Given a resolution $\dots \to X_1 \to X_0 \xrightarrow{\epsilon} C \to 0$, the induced sequence
$\Hom(X_1,A) \leftarrow \Hom(X_0,A) \xleftarrow{\epsilon^*} \Hom(C,A) \leftarrow 0$
is exact \cite[Theorem II.6.1.]{maclane}.
Hence $\ker (\Hom(X_0,A)\to \Hom(X_1,A)) \cong \Image \epsilon^*$ and therefore $\epsilon^*: \Hom(C,A) = \Ext{0}(C,A) \cong H^0(X,A)$.
For convenience we will denote this isomorphism by $\zeta$.