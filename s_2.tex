\section{Extensions and Cohomology of Groups}
\label{s:extensionsandcohomologyofgroups}
Let $G, B, \Pi$ be groups.
An exact sequence 
\begin{equation} 
\label{diag:groupextension} 
\begin{tikzcd}
E: 0 \arrow{r}& G \arrow{r}{\chi} & B \arrow{r}{\sigma} &\Pi \arrow{r}& 1
\end{tikzcd}
\end{equation}
is called a group extension of $G$ by $\Pi$.
For convenience $G,B$ are denoted as additive groups and $\Pi$ as a multiplicative group.
We define a homomorphism 
$\theta:B \to \Aut(G)$ 
via conjugation
\[
\chi (\theta(b)g) = b + \chi g - b 
\]
for $b\in B$ and $g\in G$.
Let $A:=G$ be an abelian group.
Then for any $b\in \ker \sigma$ by exactness $b=\chi a$ for some $a\in A$, hence $\theta(b) = 1_{A}$.
Therefore $\ker \sigma \subset \ker \theta$.
Thus we can define
$\phi : \Pi \to \ Aut(A)$ via the universal property of the quotient as the unique map such that $\phi\sigma = \theta$.
Call $\phi$ the operators of the extension $E$.
We have the following equality
\begin{equation}
\label{eq:operatorsphi}
\chi (\phi(\sigma(b))a) = b + \chi a - b
\end{equation}
for $a\in A$ and $b\in B$.
% Define Morphism $\Gamma: E \to E'$ via commutative diagram.
A morphism $\Gamma: E \to E'$ is a tuple of group homomorphisms $(\alpha,\beta,\gamma)$ such that the diagram
\begin{equation*}
%\label{diag:group_extension_morphism}
\begin{tikzcd}
E: \arrow{d}{\Gamma}&
A\arrow{r}\arrow{d}{\alpha}&
B \arrow{r}\arrow{d}{\beta}&
\Pi \arrow{d}{\gamma}
\\
E': &
A'\arrow{r} &
B'\arrow{r} &
\Pi'
\end{tikzcd}
\end{equation*}
commutes.
% Define congruence $E \equiv E'$ via morphism $(1_A,\beta,1_{\Pi})$
% Note that congruent extennsions have the same operators.
Two group extensions $E, E'$ are congruent if there is a morphism of the form $(1_A,\beta,1_{\Pi}):E \to E'$.
We use the familiar notation $E\equiv E'$ for congruent extensions.
As in the case of $\Ext{}$ we may speak of congruence classes of group extensions because $\beta$ is an isomorphism via the Short 5 Lemma.
\begin{lemma}
Congruent extensions have the same operators.
\end{lemma}
\begin{proof}
Consider extensions $E=(\chi,\sigma):A\to B\to \Pi$ with operators $\phi$ and $E'=(\chi',\sigma'):A\to B'\to \Pi$ with operators $\phi'$.
Assume $E$ and $E'$ are congruent, so there is a morphism $(1_A,\beta,1_{\Pi}):E \to E'$.
Let $b\in B$ and set $\sigma b = x$.
By commutativity $\sigma' \beta b = \sigma b = x$ and $\beta \chi = \chi'$.
By the definition of operators for $a\in A$ we have
\begin{align*}
\beta \chi (\phi(\sigma b)a) &= \beta b + \beta \chi a - \beta b \\
\Longrightarrow \; \beta \chi (\phi(x)a) &= \chi' (\phi'(\sigma' \beta b)a)\\
\Longrightarrow \;\;\, \chi' (\phi(x)a) &= \chi' (\phi'(x)a).
\end{align*}
Because $\chi'$ is injective the operators $\phi$ and $\phi'$ are equal.
\end{proof}
Denote the set of congruence classes of  group extensions of an abelian group $A$ by any group $\Pi$ with operators $\phi$ by $\Opext(\Pi,A,\phi)$.
\subsection{Group Ring}
In this subsection we follow \cite[Chapter 3.2.]{milies}.
Let $\Pi$ be any multiplicative group.
Let $\mathbb{Z}(\Pi)$ denote the set of all formal linear combinations
$\alpha = \sum_{x\in\Pi} a_x x$ where $a_x \in \mathbb{Z}$ with $a_x = 0$ for all but finitely many $x$.
We define the sum of two elements in $\mathbb{Z}(\Pi)$ by
\[
\left(\sum_{x\in\Pi} a_x x\right) + \left(\sum_{y\in\Pi} b_y y\right) = \sum_{x\in\Pi} (a_x + b_x) x.
\]
We define the product of two elements $\alpha=\sum_{x\in\Pi} a_x x$ and $\beta= \sum_{y\in\Pi} b_y y$ by
\[
\alpha\beta = \sum_{x,y\in\Pi} (a_x b_y) xy.
\]
These operations induce a ring structure on $\mathbb{Z}(\Pi)$.
A module over $\mathbb{Z}(\Pi)$ is called $\Pi$-module.
We want to show that a group homomorphism $\Pi\to Aut(A)$ for an abelian group $A$ defines a unique $\Pi$-module structure on $A$.

Define an embedding $\iota:\Pi \to \mathbb{Z}(\Pi)$ by $\iota(x)=1 \cdot x$.
Note that $\iota\Pi$ is a basis of $\mathbb{Z}(\Pi)$.

\begin{lemma}\cite[Proposition IV.1.1.]{maclane}
\label{lemma:universalproperty_groupring}
Let $\mu:\Pi \to R$ be a function with $\mu(1)=1$ and $\mu(xy)=\mu(x)\mu(y)$.
There is a unique ring homomorphism $\rho:\mathbb{Z}(\Pi)\to R$ such that the diagram
\[
\begin{tikzcd}
& \mathbb{Z}(\Pi) 
	\arrow{d}{\rho}
\\
\Pi 
	\arrow{ur}{\iota}
	\arrow{r}{\mu}
& R
\end{tikzcd}
\]
commutes.
\end{lemma}
\begin{proof}
Define $\rho(\sum_{x\in\Pi}a_x x):= \sum_{x\in\Pi}a_x \mu(x)$.
It is easy to check that this is a ring homomorphism and that the diagram commutes.
Since $\iota\Pi$ is a basis, $\rho$ is unique.
\end{proof}

\begin{lemma} \cite[Proposition IV.1.2.]{maclane}
Let $A$ be an abelian group.
A group homomorphism $\phi: \Pi \to Aut(A)$ gives $A$ a unique $\Pi$-module structure. 
\end{lemma}
\begin{proof}
The group homomorphisms $A \to A$ form a ring $End(A)$.
The automorphisms $Aut(A)$ are a subset of $End(A)$.
Extending the range of $\phi$ to $End(A)$ allows us to apply Lemma \ref{lemma:universalproperty_groupring}.
Hence $\phi$ induces a $\Pi$-module structure on $A$.
We say $A$ is a $\Pi$-module with operators $\phi$.
\end{proof}
%--------------------------
\subsection{Factor Sets} 
\label{ss:factorsets}
From here on we identify $A$ with $\chi A \subset B$.
For an extension $E$ as in \eqref{diag:groupextension} we choose a set function
$u:\Pi \to B$ such that $\sigma u = 1_{\Pi}$ and $u(1)=0$.
We call $u$ representatives.
Notice that $\sigma$ being onto ensure the existence of representatives.
We use the notation $xa:=\phi(x)a$.
The equation \eqref{eq:operatorsphi} with $b=u(x)$ then becomes
\begin{equation}
\label{eq:representatives_and_A}
u(x) + a = xa + u(x).
\end{equation}
Since $A = \ker\sigma$ we conclude that each coset $b + A$ in $B$ contains exactly one $u(x)$.
For suppose $u(x) = u(x') + a$ then $x = \sigma u (x) = \sigma (u(x')+ a) = \sigma u(x') + \sigma(a) = x'$.

Because $\sigma u (xy) = xy = \sigma u(x) \sigma u(y) = \sigma (u(x)+u(y))$ there exist unique elements $f(x,y)\in A$ such that $u(x) + u(y) = f(x,y) + u(xy)$.
Call $f$ a factor set of the extension $E$.

The described procedure of assigning a factor set to a given group extension will be used in Theorem \ref{thm:opext}.

The addition in $B$ is determined by $u$ and $f$ in the following way:
Every element in $B$ can be written uniquely as $a + u(x)$ for $a\in A$ and $x \in \Pi$.
For elements $a + u(x)$ and $a_1 + u(y)$ in $B$ we calculate their sum
\begin{equation*}
(a + u(x)) + (a_1 + u(y)) = a + x a_1 + u(x) + u(y) = a + x a_1 + f(x,y) + u(xy)
\end{equation*}
by using \eqref{eq:representatives_and_A} and the definition of $f$.

%%
%%Lemma needed for the main Theoremof this section
%%



\subsection{Opext and the 2-dimensional Cohomology Group}
\label{ss:opext}
We define $Z^2_\phi (\Pi,A)$ to be the set of functions $f:\Pi \times \Pi \to A$ such that the conditions
\begin{equation}
\label{eq:normalization}
f(x,1) = 0 = f(1,y)
\end{equation}
\begin{equation}
\label{eq:factorsetequality}
xf(y,z)+f(x,yz) = f(x,y) + f(xy,z)
\end{equation}
are satisfied for all $x,y,z\in \Pi$.
Denote by $B^2_\phi (\Pi,A)$ the subset of $Z^2_\phi (\Pi,A)$ containing all functions $\delta g$ of the form 
\begin{equation}
\label{eq:deltag}
\delta g (x,y) := xg(y)-g(xy)+g(x)
\end{equation}
for some function $g:\Pi \to A$ with $g(1)=0$.
We define an operation $(f + f') (x,y)= f(x,y) + f'(x,y)$.
Together with this operation $Z^2_\phi (\Pi,A)$ is an abelian group with $B^2_\phi (\Pi,A)$ being a subgroup. %abelian group => subgroups are normal
\begin{definition}
We define the 2-dimensional cohomology group as the quotient 
\[
H^2_\phi (\Pi,A) = Z^2_\phi (\Pi,A) / B^2_\phi (\Pi,A).
\]
\end{definition}
\begin{lemma}
Factor sets satisfy conditions \eqref{eq:normalization} and \eqref{eq:factorsetequality}.
\end{lemma}
\begin{proof}
It is easy to see that the equation \eqref{eq:normalization} holds for factor sets.
Let $f$ be a factor set of the extension $A\to B \to \Pi$.
Using the addition described in subsection \ref{ss:factorsets} we calculate
\begin{align*}
\text{1.\;}(u(x)+  u(y)) + u(z)  &= (f(x,y)+u(xy))+u(z) \\&= f(xy) + f(xy,z) + u(xyz) \\
\text{2.\;} u(x)+ (u(y)  + u(z)) &= u(x)+(f(y,z) + u(yz)) \\&= xf(y,z) + f(x,yz) + u(xyz)
\end{align*}
Since addition in $A$ is associative 1 and 2 are equal.
So equation \eqref{eq:factorsetequality} is satisfied.
\end{proof}
When we assigned factor sets to group extensions it involved a choice of representatives.
To prove Theorem \eqref{thm:opext} we will need
\begin{lemma}
\label{lemma:factorsetwelldefined}
The factor set of a group extension of $A$ by $\Pi$ with operators $\phi$ is well defined modulo $B_\phi^2(\Pi,A)$.
\end{lemma}
\begin{proof}
Let $u,u':\Pi\to B$ be representatives.
By definition $\sigma u (x)= \sigma u' (x) = x$ for all $x\in\Pi$.
Therefore $u(x)$ and $u'(x)$ lie in the same coset of $A$ in $B$ and for all $x\in\Pi$ we can choose some set function
$g: \Pi \to A$ such that
$u'(x) = g(x) + u(x)$.
Using \eqref{eq:representatives_and_A} we get
\begin{align*}
u'(x) + u'(y) 
&= g(x)+u(x)+g(y)+u(y) \\
&= g(x)+xg(y)+u(x)+u(y) \\
&= g(x)+xg(y)+f(x,y)+u(xy) \\  
&= g(x)+xg(y)+f(x,y)-g(xy)+u'(xy) \\
&= xg(y)-g(xy)+g(x)+f(x,y)+u'(xy) \\
&= \delta g (x,y) +f(x,y)+u'(xy)
\end{align*}
where $f$ is the factor set for representatives $u$ and $\delta g$ as defined in \eqref{eq:deltag}.
So we can define the new factor set as
\[
f'(x,y) = \delta g (x,y) +f(x,y)
\]
with $\delta g\in B_\phi^2$.
\end{proof}
\begin{theorem}
\label{thm:opext}
Assigning a factor set to a congruence class of group extensions yields a bijection
\[\omega: \Opext(\Pi,A,\phi) \to H^2_\phi(\Pi,A)\]
modulo $B^2_\phi(\Pi,A)$.
\end{theorem}
\begin{proof}
Given a congruence class $\tau\in \Opext(\Pi,A,\phi)$ choose a representative $E\in \tau$.
Now construct a factor set as described in subsection \ref{ss:factorsets}.
Define $\omega \tau := f+B^2_\phi(\Pi,A)$.
%well defined
It is easy to see that congruent extensions have the same factor sets.
Together with Lemma \ref{lemma:factorsetwelldefined}, this shows that $\omega$ is well defined.

%injective
To show that it is injective, let $E:A\to B \to \Pi$ and $E':A\to B' \to \Pi$ be two group extensions.
Choose representatives $u$ and $u'$ with factor sets $f$ and $f'$.
Assume $f' - f = \delta g$ for some set function $g: \Pi \to A$ satisfying $g(1)=0$.
Choosing representatives $g(x) + u'(x)$ for $E'$ shows that $f$ is factor set for $E'$.
As representatives and factor set determine the addition of $B$ and $B'$, the extensions are congruent.

%surjective
Lemma \ref{lemma:factorsetsareonto} shows surjectivity.
\end{proof}
\begin{lemma}
\label{lemma:factorsetsareonto}
Every $f\in Z_\phi^2$ is factor set of some group extension $E$.
\end{lemma}
\begin{proof}
Define $B:=A\times\Pi$.
The operation 
\[
(a,x)+(a_1,y) := (a+x a_1+f(x,y),xy)
\]
defined for elements $(a,x), (a_1,y)\in B$ induces a group structure on $B$.
Associativity is shown by
\begin{align*}
((a,x)+(a_1,y))+(a_2,z) 
&= (a+x a_1+f(x,y),xy) + (a_2,z)\\
&= (a+x a_1+f(x,y) + xya_2 + f(xy,z),xyz)\\
&=(a+x(a_1+y a_2+f(y,z))+f(x,yz),xyz) \\
&=(a,x)+(a_1+y a_2+f(y,z),yz)\\
&=(a,x)+((a_1,y)+(a_2,z)).
\end{align*}
The third equation holds because of \eqref{eq:factorsetequality}.
The short exact sequence $(\chi,\sigma): A\to B\to\Pi$ with $\chi$ inclusion and $\sigma$ projection has representatives $u(x)=(0,x)$ and factor set $f$.
\end{proof}

%define semi-direct product
Let $A$ be a $\Pi$-module.
The semi-direct product $A\times_\phi \Pi$ is defined as the set $A\times\Pi$ together with the addition
\[
(a,x)+(a_1,y) := (a + x a_1, x y) 
\]
for $(a,x), (a_1,y) \in A\times \Pi$.
The neutral element is $(0,1)$ and the inverse of an element $(a,x)$ is given by $-(a,x):=(x^{-1}(-a),x^{-1})$.
Let $\iota:A \to A \times_\phi \Pi$ denote the inclusion and $\rho:A \times_\phi \Pi \to \Pi$ the projection.
An extension that is congruent to $(\iota, \rho)$ is called semi-direct product extension.

The proof of Lemma \ref{lemma:factorsetsareonto} shows that the image of the zero element of $H_\phi^2(\Pi,A)$ is the semi-direct product extension.
\subsection{Bar Resolution}
\label{ss:barresolution}
Given a group $\Pi$ we will construct a chain complex $B(\mathbb{Z}(\Pi))$ of free $\Pi$-modules.
%define $B_n$ 
\begin{definition}
Let $n \geq 0$.
Define $B_n(\mathbb{Z}(\Pi))$ to be the free $\Pi$-module generated by all tuples $[x_1|x_2|\dots|x_n]$ with $x_i \in \Pi$ and $x_i \neq 1$ for $1\leq i \leq n$.
\end{definition}
Notation wise we set $[x_1|\dots|x_n]=0$ if any $x_i = 1$.
%define $\epsilon$
The module $B_0$ is generated by a single generator $[\;]$.
Regard $\mathbb{Z}$ as trivial $\Pi$-module, that is $\alpha m = m$ for all $\alpha \in \mathbb{Z}(\Pi)$ and $m\in \mathbb{Z}$.
We have a module homomorphism $\epsilon: B_0 \to \mathbb{Z}$ defined by $\epsilon [\;] := 1$.
%define $\bdd{}$
Define module homomorphisms $\bdd{}:B_n \to B_{n-1}$ by
\begin{multline*}
\bdd{}[x_1|\dots|x_n] 
:= x_1[x_2|\dots|x_n]
+\sum_{i=1}^{n-1}(-1)^i [x_1|\dots|x_i x_{i+1}|\dots|x_n]+ \\ 
+(-1)^n [x_1|\dots|x_{n-1}]
\end{multline*}
for $n > 0$.
The definition includes the case where $x_j=1$ for some $1\leq j \leq n$.
%define $s$
Regarding the $B_n$ as abelian groups generated by $x[x_1|\dots|x_n]$ we define group homomorphisms $s_{-1}: \mathbb{Z} \to B_0$ and $s_n: B_n \to B_{n+1}$ by
\begin{equation*}
s_{-1} 1 := [\;]\quad \text{and} \quad s_n x[x_1|\dots|x_n] := [x|x_1|\dots|x_n].
\end{equation*}
\begin{lemma} %$s$ is contracting homotopy of $B$
\label{lemma:s_contracting_homotopy}
The equations
\begin{equation}
\label{eq:s_contracting_homotopy}
\epsilon s_{-1} = 1_\mathbb{Z}, \quad
\bdd{} s_0 + s_{-1}\epsilon = 1_{B_0}, \quad
\bdd{}s_n + s_{n-1}\bdd{} = 1_{B_n}
\end{equation}
are satisfied.
\end{lemma}
\begin{proof}
The first equation is clear.
We compute the two summands in the second equation:
\begin{align*}
1.\quad & \bdd{} s_0 x[\;] = \bdd{}[x] = x [\;] - [\;] \\
2.\quad & s_{-1}\epsilon x[\;] = s_{-1} x 1 = s_{-1} 1 = [\;]
\end{align*}
Adding 1 and 2 proves the second equation.
Now we calculate the summands of the remaining equation.
\begin{equation*}
\begin{split}
1.\quad \bdd{}s_n x_0[x_1|\dots|x_n]
&= \bdd{} [x_0|x_1|\dots|x_n] \\
&= x_0[x_1|\dots|x_n]\\
&\quad + \sum_{i=0}^{n-1}(-1)^{i+1} [x_0|x_1|\dots|x_i x_{i+1}|\dots|x_n]\\
&\quad + (-1)^{n+1} [x_0|x_1|\dots|x_{n-1}])
\end{split}
\end{equation*}
\begin{equation*}
\begin{split}
2.\quad s_{n-1}\bdd{} x_0[x_1|\dots|x_n]
&= s_{n-1} x_0 (x_1[x_2|\dots|x_n] \\
&\quad +\sum_{i=1}^{n-1}(-1)^i [x_1|\dots|x_i x_{i+1}|\dots|x_n] \\
&\quad +(-1)^n [x_1|\dots|x_{n-1}]) \\
&= [x_0 x_1|x_2|\dots|x_n] \\
&\quad +\sum_{i=1}^{n-1}(-1)^i [x_0|x_1|\dots|x_i x_{i+1}|\dots|x_n] \\
&\quad +(-1)^n [x_0|x_1|\dots|x_{n-1}]) \\
&= \sum_{i=0}^{n-1}(-1)^i [x_0|x_1|\dots|x_i x_{i+1}|\dots|x_n] \\
&\quad +(-1)^n [x_0|x_1|\dots|x_{n-1}])
\end{split}
\end{equation*}
The two sums cancel, as the $[x_0|x_1|\dots|x_{n-1}]$ terms do.
This proves the last equation.
\end{proof}
\begin{lemma}
\label{lemma:B_complex}
$(B_n,\bdd{n})$ is a complex.
\end{lemma}
\begin{proof}
Observe that 
$\epsilon\bdd{1}([x])=\epsilon (x[\;]-[\;]) = 0$.
We can rewrite the third equation in \eqref{eq:s_contracting_homotopy} as
$\bdd{n+1}s_n = 1- s_{n-1}$.
Applying this twice results in
\begin{multline}
\label{eq:B_complex}
\bdd{n}\bdd{n+1}s_n
= \bdd{n}(1- s_{n-1}\bdd{n})
= \bdd{n}-\bdd{n}s_{n-1}\bdd{n} \\
= \bdd{n}-(1- s_{n-2}\bdd{n-1})\bdd{n}
= s_{n-2}\bdd{n-1}\bdd{n}.
\end{multline}
The module $B_{n+1}$ is generated by $s_{n} B_{n}$.
By induction it follows that $\bdd{n}\bdd{n+1} = 0$.
\end{proof}

Let $x\in \ker\bdd{n}$. By \eqref{eq:s_contracting_homotopy} $\bdd{n+1}s_nx=x$. This means $x\in \Image\bdd{n+1}$.
Therefore $\ker \bdd{n} \subset \bdd{n+1} B_{n+1}$ and $B(\mathbb{Z}(\Pi)) := (B_n,\bdd{n})$ is a resolution of $\mathbb{Z}$.

Define the $n$-dimensional cohomology group of $\Pi$ with coefficients in the $\Pi$-module $A$ by $H^n(\Pi,A) := H^n(B(\mathbb{Z}(\Pi)),A)$.

An element of $\Hom(B_2,A)$ is a module homomorphisms $f:B_2 \to A$.
It is determined by the images of the module generators $[x|y]$.
Hence we can view $f$ as a function from $\Pi \times \Pi$ to $A$ with $f(x,1) = f(1,y) = 0$.
If $f$ is a cocycle, it satisfies \eqref{eq:factorsetequality}.
Thus $f$ can be viewed as an element of $Z^2_\phi(\Pi,A)$.
The subgroup $\bdd{}B_3$ can be identified with $B_\phi^2(\Pi, A)$ in the same manner.
This means that given a $\Pi$-module $A$, with module structure recorded by the operators $\phi$, the assigned groups $H^2_\phi(\Pi, A)$ and $H^2(\Pi,A)$ are isomorphic.

The $n$-dimensional cohomology of groups is a special case of $\Ext{n}$.
This is shown by

\begin{theorem}\cite[Corollary IV.5.2.]{maclane}
\label{theorem:extandcohom}
Let $A$ be a $\Pi$-module.
There is an isomorphism
\[\Ext{n}_{\mathbb{Z}(\Pi)}(\mathbb{Z},A)\cong H^{n}(\Pi,A),\]
which is natural in $A$.
\end{theorem}
\begin{proof}
We proved that $B(\mathbb{Z}(\Pi))$ is a resolution of $\mathbb{Z}$ as a trivial $\Pi$-module and the $B_n$ are free by construction.
Since free modules are projective, the isomorphism and its naturality follow from Theorem \ref{thm:zeta}.
\end{proof}
We saw in subsection \ref{ss:alongexactsequenceforextn} that for any short exact sequence $0 \to A \to B \to C \to 0$ there is a long exact sequence for $\Ext{n}$.
Theorem \ref{theorem:extandcohom} therefore yields a long exact sequence
\begin{equation*}
\dots \to H^n(\Pi, A) \to H^n(\Pi, B) \to H^n(\Pi, C) \to H^{n
+1}(\Pi,A) \to \dots
\end{equation*}
We end this section with two calculations using Theorem \ref{theorem:extandcohom}.
\begin{application}\cite[Chapter IV.7.]{maclane}
\label{app}
Let $A$ be an abelian group and $m$ a positive integer.
We denote the cyclic group of order $m$ by $C_m$ and its generator by $t$.
We calculate $H^n(C_m, A)$ for $n>0$.
\end{application}
First we construct a projective resolution of $\mathbb{Z}$ as a $C_m$-module.
Let $\Gamma$ denote the group ring $\mathbb{Z}(C_m)$.
The elements
\[
N := 1 + t + \cdots + t^{m-1}\quad \text{and} \quad D := t - 1
\]
induce maps $N_*, D_*: \Gamma \to \Gamma$ via multiplication.
It is easy to see that $ND = 0$.
Therefore
\[
\begin{tikzcd}
W : \quad \cdots \arrow{r}{N_*} & \Gamma \arrow{r}{D_*} & \Gamma \arrow{r}{N_*} & \Gamma \arrow{r}{D_*} & \Gamma .
\end{tikzcd}
\]
defines a complex.
Furthermore the module homomorphism $\epsilon: \Gamma \to \mathbb{Z}$ defined by $\epsilon \left( \sum a_i t_i \right) =\sum a_i$ sends all elements of the form $Du$ for $u\in \Gamma$ to zero.
Therefore
\begin{equation*}
\begin{tikzcd}
(W,\epsilon): \quad \cdots \arrow{r}{N_*} & \Gamma \arrow{r}{D_*} & \Gamma \arrow{r}{N_*} & \Gamma \arrow{r}{D_*} & \Gamma \arrow{r}{\epsilon}& \mathbb{Z} \arrow{r}& 0
\end{tikzcd}
\end{equation*}
is a complex over $\mathbb{Z}$.
To show exactness we have to look at elements $u:=\sum_{i=0}^{m-1}a_i t^i\in \Gamma$.
\begin{enumerate}
\item
Suppose $Du=0$.
Then $Du = \sum_{i=0}^{m-1}(a_{i-1}-a_{i})t^i = 0$ (with $a_{-1}:=a_{m-1}$) and therefore $a_i = a_{i-1}$ for all $i$.
So we can write $u=a_0 N$.
\item
Suppose $Nu=0$.
Then $Nu = \sum_j \left( \sum_i a_i\right)t^j = 0$ which implies $\sum_i a_i = 0$.
Hence we can write $u = -D (a_0 + (a_1+a_0)t + \cdots + (a_{m-1}+\cdots+a_0)t^{m-1})$.
\item
Suppose $\epsilon u = 0$. Then $\sum_i a_i = 0$ and as above we see that we can write $u=  -D (a_0 + (a_1+a_0)t + \cdots + (a_{m-1}+\cdots+a_0)t^{m-1})$.
\end{enumerate}
We proved that $(W,\epsilon)$ is a resolution over $\mathbb{Z}$.
The ring $\Gamma$ is a projective module over itself.
So by Theorem \ref{thm:zeta} we can compute $\Ext{n}$ using $(W,\epsilon)$.

Note that $\Hom(\Gamma,A)\cong A$ since any homomorphism $f:\Gamma \to A$ is determined by $f(1)\in A$. 
Thus applying $\Hom(\_,A)$ yields the sequence
\begin{equation*}
\dots \xleftarrow{N^*} A \xleftarrow{D^*} A \xleftarrow{N^*} A \xleftarrow{D^*} A.
\end{equation*}
Calculating the homology of this complex results in
\begin{gather*}
H^{2n}(C_m, A) = [a|ta = a]/N^* A, n \geq 0\\
H^{2n+1}(C_m, A) =[a|Na = 0]/D^* A, n > 0.
\end{gather*}
\begin{example}
With the same situation as in Application \ref{app}, but $A$ a trivial $C_m$-module, the cohomology groups are
\begin{gather*}
H^{2n}(C_m, A) = [a\in A|ta = a]/N^* A = A / mA, n \geq 0\\
H^{2n+1}(C_m, A) =[a \in A|Na = 0]/D^* A = [a|\text{order of $a$ divides $m$}], n > 0.
\end{gather*}
\end{example}
\begin{proof}
All elements of $A$ are invariant under $C_m$.
The image of $N^*$ is all elements of the form $ma$, $a\in A$.
The kernel of $N^*$ are all elements with order dividing $m$.
The image of elements $a\in A$ under $D^*$ is $(1-t)a = a - a = 0$.
Now the results follow by our calculations in Application \ref{app}
\end{proof}