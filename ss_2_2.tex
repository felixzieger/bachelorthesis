\subsection{Bar Resolution}
\label{ss:barresolution}
Given a group $\Pi$ we will construct a chain complex $B(\mathbb{Z}(\Pi))$ of free $\Pi$-modules.
%define $B_n$ 
\begin{definition}
Let $n \geq 0$.
Define $B_n(\mathbb{Z}(\Pi))$ to be the free $\Pi$-module generated by all tuples $[x_1|x_2|\dots|x_n]$ with $x_i \in \Pi$ and $x_i \neq 1$ for $1\leq i \leq n$.
\end{definition}
Notation wise we set $[x_1|\dots|x_n]=0$ if any $x_i = 1$.
%define $\epsilon$
The module $B_0$ is generated by a single generator $[\;]$.
Regard $\mathbb{Z}$ as trivial $\Pi$-module, that is $\alpha m = m$ for all $\alpha \in \mathbb{Z}(\Pi)$ and $m\in \mathbb{Z}$.
We have a module homomorphism $\epsilon: B_0 \to \mathbb{Z}$ defined by $\epsilon [\;] := 1$.
%define $\bdd{}$
Define module homomorphisms $\bdd{}:B_n \to B_{n-1}$ by
\begin{multline*}
\bdd{}[x_1|\dots|x_n] 
:= x_1[x_2|\dots|x_n]
+\sum_{i=1}^{n-1}(-1)^i [x_1|\dots|x_i x_{i+1}|\dots|x_n]+ \\ 
+(-1)^n [x_1|\dots|x_{n-1}]
\end{multline*}
for $n > 0$.
The definition includes the case where $x_j=1$ for some $1\leq j \leq n$.
%define $s$
Regarding the $B_n$ as abelian groups generated by $x[x_1|\dots|x_n]$ we define group homomorphisms $s_{-1}: \mathbb{Z} \to B_0$ and $s_n: B_n \to B_{n+1}$ by
\begin{equation*}
s_{-1} 1 := [\;]\quad \text{and} \quad s_n x[x_1|\dots|x_n] := [x|x_1|\dots|x_n].
\end{equation*}
\begin{lemma} %$s$ is contracting homotopy of $B$
\label{lemma:s_contracting_homotopy}
The equations
\begin{equation}
\label{eq:s_contracting_homotopy}
\epsilon s_{-1} = 1_\mathbb{Z}, \quad
\bdd{} s_0 + s_{-1}\epsilon = 1_{B_0}, \quad
\bdd{}s_n + s_{n-1}\bdd{} = 1_{B_n}
\end{equation}
are satisfied.
\end{lemma}
\begin{proof}
The first equation is clear.
We compute the two summands in the second equation:
\begin{align*}
1.\quad & \bdd{} s_0 x[\;] = \bdd{}[x] = x [\;] - [\;] \\
2.\quad & s_{-1}\epsilon x[\;] = s_{-1} x 1 = s_{-1} 1 = [\;]
\end{align*}
Adding 1 and 2 proves the second equation.
Now we calculate the summands of the remaining equation.
\begin{equation*}
\begin{split}
1.\quad \bdd{}s_n x_0[x_1|\dots|x_n]
&= \bdd{} [x_0|x_1|\dots|x_n] \\
&= x_0[x_1|\dots|x_n]\\
&\quad + \sum_{i=0}^{n-1}(-1)^{i+1} [x_0|x_1|\dots|x_i x_{i+1}|\dots|x_n]\\
&\quad + (-1)^{n+1} [x_0|x_1|\dots|x_{n-1}])
\end{split}
\end{equation*}
\begin{equation*}
\begin{split}
2.\quad s_{n-1}\bdd{} x_0[x_1|\dots|x_n]
&= s_{n-1} x_0 (x_1[x_2|\dots|x_n] \\
&\quad +\sum_{i=1}^{n-1}(-1)^i [x_1|\dots|x_i x_{i+1}|\dots|x_n] \\
&\quad +(-1)^n [x_1|\dots|x_{n-1}]) \\
&= [x_0 x_1|x_2|\dots|x_n] \\
&\quad +\sum_{i=1}^{n-1}(-1)^i [x_0|x_1|\dots|x_i x_{i+1}|\dots|x_n] \\
&\quad +(-1)^n [x_0|x_1|\dots|x_{n-1}]) \\
&= \sum_{i=0}^{n-1}(-1)^i [x_0|x_1|\dots|x_i x_{i+1}|\dots|x_n] \\
&\quad +(-1)^n [x_0|x_1|\dots|x_{n-1}])
\end{split}
\end{equation*}
The two sums cancel, as the $[x_0|x_1|\dots|x_{n-1}]$ terms do.
This proves the last equation.
\end{proof}
\begin{lemma}
\label{lemma:B_complex}
$(B_n,\bdd{n})$ is a complex.
\end{lemma}
\begin{proof}
Observe that 
$\epsilon\bdd{1}([x])=\epsilon (x[\;]-[\;]) = 0$.
We can rewrite the third equation in \eqref{eq:s_contracting_homotopy} as
$\bdd{n+1}s_n = 1- s_{n-1}$.
Applying this twice results in
\begin{multline}
\label{eq:B_complex}
\bdd{n}\bdd{n+1}s_n
= \bdd{n}(1- s_{n-1}\bdd{n})
= \bdd{n}-\bdd{n}s_{n-1}\bdd{n} \\
= \bdd{n}-(1- s_{n-2}\bdd{n-1})\bdd{n}
= s_{n-2}\bdd{n-1}\bdd{n}.
\end{multline}
The module $B_{n+1}$ is generated by $s_{n} B_{n}$.
By induction it follows that $\bdd{n}\bdd{n+1} = 0$.
\end{proof}

Let $x\in \ker\bdd{n}$. By \eqref{eq:s_contracting_homotopy} $\bdd{n+1}s_nx=x$. This means $x\in \Image\bdd{n+1}$.
Therefore $\ker \bdd{n} \subset \bdd{n+1} B_{n+1}$ and $B(\mathbb{Z}(\Pi)) := (B_n,\bdd{n})$ is a resolution of $\mathbb{Z}$.

Define the $n$-dimensional cohomology group of $\Pi$ with coefficients in the $\Pi$-module $A$ by $H^n(\Pi,A) := H^n(B(\mathbb{Z}(\Pi)),A)$.

An element of $\Hom(B_2,A)$ is a module homomorphisms $f:B_2 \to A$.
It is determined by the images of the module generators $[x|y]$.
Hence we can view $f$ as a function from $\Pi \times \Pi$ to $A$ with $f(x,1) = f(1,y) = 0$.
If $f$ is a cocycle, it satisfies \eqref{eq:factorsetequality}.
Thus $f$ can be viewed as an element of $Z^2_\phi(\Pi,A)$.
The subgroup $\bdd{}B_3$ can be identified with $B_\phi^2(\Pi, A)$ in the same manner.
This means that given a $\Pi$-module $A$, with module structure recorded by the operators $\phi$, the assigned groups $H^2_\phi(\Pi, A)$ and $H^2(\Pi,A)$ are isomorphic.

The $n$-dimensional cohomology of groups is a special case of $\Ext{n}$.
This is shown by

\begin{theorem}\cite[Corollary IV.5.2.]{maclane}
\label{theorem:extandcohom}
Let $A$ be a $\Pi$-module.
There is an isomorphism
\[\Ext{n}_{\mathbb{Z}(\Pi)}(\mathbb{Z},A)\cong H^{n}(\Pi,A),\]
which is natural in $A$.
\end{theorem}
\begin{proof}
We proved that $B(\mathbb{Z}(\Pi))$ is a resolution of $\mathbb{Z}$ as a trivial $\Pi$-module and the $B_n$ are free by construction.
Since free modules are projective, the isomorphism and its naturality follow from Theorem \ref{thm:zeta}.
\end{proof}
We saw in subsection \ref{ss:alongexactsequenceforextn} that for any short exact sequence $0 \to A \to B \to C \to 0$ there is a long exact sequence for $\Ext{n}$.
Theorem \ref{theorem:extandcohom} therefore yields a long exact sequence
\begin{equation*}
\dots \to H^n(\Pi, A) \to H^n(\Pi, B) \to H^n(\Pi, C) \to H^{n
+1}(\Pi,A) \to \dots
\end{equation*}
We end this section with two calculations using Theorem \ref{theorem:extandcohom}.
\begin{application}\cite[Chapter IV.7.]{maclane}
\label{app}
Let $A$ be an abelian group and $m$ a positive integer.
We denote the cyclic group of order $m$ by $C_m$ and its generator by $t$.
We calculate $H^n(C_m, A)$ for $n>0$.
\end{application}
First we construct a projective resolution of $\mathbb{Z}$ as a $C_m$-module.
Let $\Gamma$ denote the group ring $\mathbb{Z}(C_m)$.
The elements
\[
N := 1 + t + \cdots + t^{m-1}\quad \text{and} \quad D := t - 1
\]
induce maps $N_*, D_*: \Gamma \to \Gamma$ via multiplication.
It is easy to see that $ND = 0$.
Therefore
\[
\begin{tikzcd}
W : \quad \cdots \arrow{r}{N_*} & \Gamma \arrow{r}{D_*} & \Gamma \arrow{r}{N_*} & \Gamma \arrow{r}{D_*} & \Gamma .
\end{tikzcd}
\]
defines a complex.
Furthermore the module homomorphism $\epsilon: \Gamma \to \mathbb{Z}$ defined by $\epsilon \left( \sum a_i t_i \right) =\sum a_i$ sends all elements of the form $Du$ for $u\in \Gamma$ to zero.
Therefore
\begin{equation*}
\begin{tikzcd}
(W,\epsilon): \quad \cdots \arrow{r}{N_*} & \Gamma \arrow{r}{D_*} & \Gamma \arrow{r}{N_*} & \Gamma \arrow{r}{D_*} & \Gamma \arrow{r}{\epsilon}& \mathbb{Z} \arrow{r}& 0
\end{tikzcd}
\end{equation*}
is a complex over $\mathbb{Z}$.
To show exactness we have to look at elements $u:=\sum_{i=0}^{m-1}a_i t^i\in \Gamma$.
\begin{enumerate}
\item
Suppose $Du=0$.
Then $Du = \sum_{i=0}^{m-1}(a_{i-1}-a_{i})t^i = 0$ (with $a_{-1}:=a_{m-1}$) and therefore $a_i = a_{i-1}$ for all $i$.
So we can write $u=a_0 N$.
\item
Suppose $Nu=0$.
Then $Nu = \sum_j \left( \sum_i a_i\right)t^j = 0$ which implies $\sum_i a_i = 0$.
Hence we can write $u = -D (a_0 + (a_1+a_0)t + \cdots + (a_{m-1}+\cdots+a_0)t^{m-1})$.
\item
Suppose $\epsilon u = 0$. Then $\sum_i a_i = 0$ and as above we see that we can write $u=  -D (a_0 + (a_1+a_0)t + \cdots + (a_{m-1}+\cdots+a_0)t^{m-1})$.
\end{enumerate}
We proved that $(W,\epsilon)$ is a resolution over $\mathbb{Z}$.
The ring $\Gamma$ is a projective module over itself.
So by Theorem \ref{thm:zeta} we can compute $\Ext{n}$ using $(W,\epsilon)$.

Note that $\Hom(\Gamma,A)\cong A$ since any homomorphism $f:\Gamma \to A$ is determined by $f(1)\in A$. 
Thus applying $\Hom(\_,A)$ yields the sequence
\begin{equation*}
\dots \xleftarrow{N^*} A \xleftarrow{D^*} A \xleftarrow{N^*} A \xleftarrow{D^*} A.
\end{equation*}
Calculating the homology of this complex results in
\begin{gather*}
H^{2n}(C_m, A) = [a|ta = a]/N^* A, n \geq 0\\
H^{2n+1}(C_m, A) =[a|Na = 0]/D^* A, n > 0.
\end{gather*}
\begin{example}
With the same situation as in Application \ref{app}, but $A$ a trivial $C_m$-module, the cohomology groups are
\begin{gather*}
H^{2n}(C_m, A) = [a\in A|ta = a]/N^* A = A / mA, n \geq 0\\
H^{2n+1}(C_m, A) =[a \in A|Na = 0]/D^* A = [a|\text{order of $a$ divides $m$}], n > 0.
\end{gather*}
\end{example}
\begin{proof}
All elements of $A$ are invariant under $C_m$.
The image of $N^*$ is all elements of the form $ma$, $a\in A$.
The kernel of $N^*$ are all elements with order dividing $m$.
The image of elements $a\in A$ under $D^*$ is $(1-t)a = a - a = 0$.
Now the results follow by our calculations in Application \ref{app}
\end{proof}