\begin{otherlanguage}{german}
\textbf{Zusammenfassung:}
In dieser Arbeit werden grundlegende Konzepte aus dem Bereich der homologischen Algebra vorgestellt.
Ziel ist es einen Bezug zwischen Gruppenerweiterungen und der $2$-dimensionalen Kohomologiegruppe herzustellen.

In Kapitel \ref{s:extensionsandresolutions} wird der Bifunktor $\Ext {n}_R$ eingeführt.
Dies geschieht mittels Erweiterungen von Moduln. 
Zur Berechnung von $\Ext{n}_R$ wird in Kapitel \ref{ss:extnandresolutions} ein Isomorphismus zwischen $\Ext{n}_R(C,A)$ und der $n$-dimensionalen Kohomologiegruppe $H^n(X,A)$ für eine projektive Auflösung $X$ des $R$-Moduls $C$ gegeben.
Außerdem wird in Kapitel \ref{ss:alongexactsequenceforextn} eine lange exakte Sequenz für $\Ext{n}_R$ konstruiert.

Kapitel \ref{s:extensionsandcohomologyofgroups} beginnt mit Erweiterungen von Gruppen.
Für eine abelsche Gruppe $A$ und eine beliebige Gruppe $\Pi$, wird eine Menge an Kongruenzklassen $Opext(\Pi,A,\phi)$ eingeführt.
Einer Erweiterung ordnen wir in Kapitel \ref{ss:factorsets} eine Funktion zu, die  Faktor System (engl. factor set) genannt wird.
Die $2$-dimensionale Kohomologiegruppe $H^2_\phi(\Pi,A)$ ist als Quotient der Menge der Faktor Systeme definiert.
In Kapitel \ref{ss:opext} bilden wir $H^2_\phi(\Pi,A)$ bijektiv auf $Opext(\Pi,A,\phi)$ ab.
Zudem wird in Kapitel \ref{ss:barresolution} eine Definition der $n$-dimensionalen Kohomologiegruppe $H^n(\Pi,A)$ mittels Auflösungen gegeben.
Die Arbeit aus dem Kapitel \ref{s:extensionsandresolutions} zeigt abschließend, dass die Kohomologie von Gruppen ein Spezialfall von $\Ext{n}_R$ mit $R=\mathbb{Z}(\Pi)$, dem Gruppenring, ist.
\end{otherlanguage}